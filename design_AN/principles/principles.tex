\section{Design principles}

\paragraph{Stochastic simulation}
In order to meet the requirements listed above,
we opted for a Gillespie-based simulator.
The Gillespie algorithm has two important features in our context:
\begin{itemize}
  \item It is a stochastic algorithm, so it naturally enables to simulate
  low-level stochastic processes.
  \item It offers a framework where an arbitrary number of reactions can
  be added.
\end{itemize}
Using the Gillespie algorithm, we can both simulate events at the molecular
level and aggregated processes.
The description level of a process simply depends on the number of reactions
that the user has chosen to represent the process.
MyCellS starts with an empty system of reactions.
The user controls what reactions to add.
Processes can easily be tuned to match a specific bacterial species.

\paragraph{Sequence-based reactions}
Standard Gillespie simulators only implement chemical reactions.
This does not meet our requirement of simulating a wide variety of processes.
For example, it is extremely tedious (nearly impossible) to simulate
translation accurately using only chemical reactions.
Take a simple molecule of species $A$ translocating along a sequence $S$
of length 100.
Here are three possible models for these translocation events:

\[
\begin{array}{c|c|c}
  \text{Model A} & \text{Model B} & \text{MyCellS}\\
  & & \\
  \begin{array}{rcl}
    S + A & \rightarrow & SA_1 \\
    SA_1 & \rightarrow & SA_2 \\
    SA_2 & \rightarrow & SA_3 \\
          & \cdots & \\
    SA_{99} & \rightarrow & SA_{100}
  \end{array}
  &
  \begin{array}{rcl}
    S + A & \rightarrow & S + A^{bound}_1 \\
    A^{bound}_1 & \rightarrow & A^{bound}_2 \\
    A^{bound}_2 & \rightarrow & A^{bound}_3 \\
          & \cdots & \\
    A^{bound}_{99} & \rightarrow & A^{bound}_{100}
  \end{array}
  &
  \begin{array}{rcl}
    S + A & \rightarrow & S + A^{bound} \text{ (Binding)} \\
    A^{bound}_i & \rightarrow & A^{bound}_{i+1} \text{ (Translocation)} \\
  \end{array}
\end{array}
\]

Subscripts show the position of $A$ along the sequence.
Model A and B use only chemical reactions, both involve duplicating
translocation events.
In practice, this implies creating thousands of reactions and chemical
species.
In model A, the sequence $S$ can only bind one chemical at a time,
which is not realistic for biological systems
(there is a high number of ribosomes per mRNA for example).
Model B enables several molecules to ``bind'' the sequence at a given time.
However, bound molecules do not interact with each other,
\textit{e.g.} it is impossible to handle sequestration of binding motifs.
In MyCellS, we define new types of reactants and reactions
for sequence-based events.
A single reaction and a single chemical species are used to represent
all translocation events.

We created a variant of the Gillespie algorithm where new types of reactants
and reactions can be plugged in.
We defined a minimal set of reactants and reactions that handles
sequence-based reactions (\textit{e.g.} binding, translation elongation).
All reactions remain low-level, enabling flexible descriptions of
complex processes using a limited number of reactions.
A lot of information that is provided as an input for these reactions comes from
standard sequence annotation.
When switching from an organism to another, the key reactions that define the
process remain the same.
Only sequence information (DNA, position of genes, promoters, etc.) and rates
need to be adapted.

\paragraph{Efficiency}
We evaluated performance by simulating protein production.
Protein production (from gene to protein) is responsible for
a large number of reactions in a bacterial cell (metabolism aside).
Simulation is completed within hours even for detailed descriptions of all
processes involved (stochastic base-by-base elongation with all cofactors).
This objective was reached by using the latest implementations
of the (exact) Gillespie algorithm.
We also tuned all new types of reactions to be nearly as efficient as chemical
reactions.

\paragraph{Modularity}
We created clear modules in MyCellS's structure.
This allows for core changes and facilitates communication with external modules.
Typical core changes involve:
\begin{itemize}
  \item Plugging in new solver variants
  (\textit{e.g.} new implementation of the  exact Gillespie algorithm,
  implementation of approximations such as $\tau$-leaping).
  \item Plugging in new reactants and reactions.
\end{itemize}
Interfaces of the modules were designed for these operations to be pure
plug-in operations (no need to change the code in existing modules).
A similar design applies for external programs.
Reactant concentrations can be modified during the course of a simulation.
This enables to plug-in arbitrary external solvers.
For example, we intend to plug-in a deterministic solver for metabolism on
MyCellS.
