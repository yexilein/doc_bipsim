
\subsection{Input/Output handling}

\subsubsection{Simulator Input}

The simulator needs the following to work:
\begin{itemize}
  \item A general input file defining simulation parameters.
  A sample file is provided were all options are described
  (\textit{e.g.} length of simulation, what to output, algorithm variants).
  One important parameter is the location of the files the simulator
  should open to read reactants, reactions and events.
  \item An arbitrary number of files where reactants, reactions and events are declared.
  The simulator solves dependencies across files, it is not necessary
  to declare reactants in the same file as or before reactions using them.
\end{itemize}

Caution:
\begin{itemize}
\item All reactants must be declared in some file with their appropriate type
(\textit{e.g.} \texttt{FreeChemical} or \texttt{BoundChemical}).
\item Multiple declarations are forbidden, a name cannot be reused.
\end{itemize}

\subsubsection{Simulator Output}

Outputs are dispatched among several files.
Their content is controlled by the user through the parameter file.
Possible output files are:
\begin{itemize}
\item A general output file logging parameters used for simulation (input files used, random seed, algorithms used, etc.).
\item A concentration file with the number of molecules over time (for the chemicals and at a time step defined in the parameter file).
\item If a \texttt{DoubleStrand} was added in the chemicals to output,
a replication file describing replication advancement of that \texttt{DoubleStrand}.
\end{itemize}
