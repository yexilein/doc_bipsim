
\subsection{Switches}

\paragraph{Input format}
\begin{verbatim}
Switch <name> <input_bound_chemical> <output_bound_chemical>
SwitchSite <chemical_sequence> <position> <switch_name>
\end{verbatim}

\texttt{Switch}es are intrinsically linked to \texttt{BoundChemical}s
but apply to specific \texttt{BoundUnit}s through \texttt{SwitchSite}s
located on \texttt{ChemicalSequence}s.
Every time an instance of \texttt{input\_bound\_chemical} steps on a switch site,
it \emph{immediately} becomes an \texttt{output\_bound\_chemical}.

For example, during transcription, an RNA polymerase (RNAP) goes through an
initiation state, then loops through several elongation states
(loading of a nucleotide and translocation).
Once it reaches a termination site represented by a \texttt{SwitchSite},
the RNAP leaves its current elongation state and enters termination state.
It stops performing polymerization reactions and typically releases the polymerization product
and unbinds from DNA.\@

A \texttt{Switch} is not considered a reaction because there is no rate associated with it
(switches are performed automatically before the solver chooses the next reaction).
We dedicate a section to these elements because they play a central role in the simulator's philosophy.
The user can use generic reactions that apply in general
(\textit{e.g.} transcription of any gene based on its sequence)
and use switches every time something more specific is needed.
As seen before, termination sites for transcription are expected to be \textit{SwitchSite}s.
Similarly, important regulation sites can be implemented using \textit{SwitchSite}s.
