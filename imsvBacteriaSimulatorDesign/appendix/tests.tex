
\section{Tests}

\subsection{Testing philosophy}

Tests are usually divided in several categories. Because of the size of the project, the program includes three types of tests: \emph{programming by contract}, \emph{unit tests}, \emph{integration tests}~\reftabp{tab:tests}. They are designed to make the program fail as rapidly as possible and help find the origin of the problem.

\newcolumntype{M}[1]{>{\centering\arraybackslash}m{#1}}
\begin{table}[!h]
  \centering
  \begin{tabular}{|c||M{0.25\textwidth}|M{0.25\textwidth}|M{0.25\textwidth}|}
    \hline
    Test type & Preconditions \newline Postconditions \newline Invariants & Unit Tests & Integration Tests \\
    \hline\hline
    Test level & Implementation details & Class interface & Systemic\\
    \hline
    Time per test & a few instructions (ns) & ms to a few seconds & seconds to several minutes\\
    \hline
    Use frequency & Permanent & Very frequent & Less frequent\\
    \hline
  \end{tabular}

  \caption{Comparisons of tests used to develop the simulator}
  \label{tab:tests}
\end{table}



\subsubsection{Programming by contract} 

These tests typically apply to attributes of classes and arguments of methods. They are usually divided into three subcategories: \emph{preconditions}, \emph{postconditions} and \emph{invariants}. They check whether the class interact correctly with the outside world, generally other classes.

\paragraph{Preconditions} Preconditions.

\subsubsection{Unit tests}

\subsubsection{Integration tests}

\subsection{Organizing and running tests}
