% ============================================
%  Article Class (This is a LaTeX2e document)  
% ============================================
\documentclass[12pt]{scrartcl}
\usepackage[english]{babel}
\usepackage{enumitem}
\usepackage[round]{natbib}
\usepackage{color}
\usepackage{array}

\newcommand\reft[2]{#1~\ref{#2}}
\newcommand\refp[2]{(#1~\ref{#2})}
\newcommand\refsect[1]{\reft{Section}{#1}}
\newcommand\refsecp[1]{\refp{Sec.}{#1}}
\newcommand\reftabt[1]{\reft{Table}{#1}}
\newcommand\reftabp[1]{\refp{Tab.}{#1}}

% ============
%  Algorithms
% ============
\usepackage{algorithm2e}
\SetKwProg{Fn}{Function}{}{}
\newcommand\refalgt[1]{\reft{Algorithm}{#1}}
\newcommand\refalgp[1]{\refp{Alg.}{#1}}

% ======
%  Math
% ======
\usepackage{amsmath}
\usepackage{amsthm}
\newtheorem{thm}{Theorem}[section]
\newtheorem{cor}[thm]{Corollary}
\newtheorem{lem}[thm]{Lemma}
\newtheorem{prop}[thm]{Proposition}
\newtheorem{property}[thm]{Property}
\theoremstyle{definition}
\newtheorem{defn}[thm]{Definition}
\newtheorem{assum}[thm]{Assumption}
\theoremstyle{remark}
\newtheorem{rem}[thm]{Remark}
\numberwithin{equation}{section}
\usepackage{amssymb}
\newcommand{\prob}[1]{\mathbb{P}\left(#1\right)}

% ====================
%  Chemical reactions
% ====================
\usepackage[version=3]{mhchem}
\newcommand{\reactionRev}[4]{#1 \ce{<=>[#3][#4]} #2}
\newcommand{\reactionIrr}[4]{#1 \ce{->[#3][#4]} #2}

% ============================
%  Figures and relative paths
% ============================
\usepackage{graphicx}
\graphicspath{{figures/}}
\usepackage{import}
\makeatletter
  \def\relativepath{\import@path}
\makeatother
\newcommand\reffigt[1]{\reft{Figure}{#1}}
\newcommand\reffigp[1]{\refp{Fig.}{#1}}

% ==========
%  Document
% ==========
\begin{document}

\title{{IMSV}: bacteria simulator design \\ Version 0.0}%
\author{M.~Dinh and S.~Fischer}%
%\address{er}%
%\thanks{zq}%
\date{\today}%
\maketitle

\newpage

\tableofcontents

\newpage

\section{Introduction}

The aim of this document is not to go into technical details of the implementation (the code is documented using Doxygen, technical details are therefore best found in the Doxygen-generated manual). Rather we wish to walk through the choices in design that have been made. The technical manual information for understanding the concept behind each class and how to use it. However it does not tell you what classes are central in the architecture and it is difficult to see at a glimpse how classes interact.

Describing global design should be more agreeable to read than the purely technical document. It helps understand how we tackled a certain number of efficiency issues (should it be speed or maintainability). The document should facilitate discussions even with non-programmers (or at least non-C++-programmers). We start by giving a global overview of the base components of the simulator, organized around reactants and reactions. A second section redescribes the same element again, but goes further into hypotheses and critical design elements. Finally, appendices are added to describe elements that have been important in the simulator development but did not fit naturally in the main document (testing strategies, utility classes, etc.).

\subimport{global/}{global.tex}
\subimport{detailed/}{detailed.tex}
\clearpage

\appendix
\subimport{appendix/}{appendix}
\clearpage

\end{document}
% ----------------------------------------------------------------
