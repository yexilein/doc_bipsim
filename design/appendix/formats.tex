
\subsection{Formats and Conventions}

\subsubsection{Input format description}

\begin{itemize}
  \item A plain word indicates a tag, that needs to be written.
  \item \texttt{<...>} indicates a variable that has to be completed with an existent element of the specified type.
  \item \texttt{[...]} indicates an optional part.
  \item \texttt{[...]\^{}\{0..n\}} indicates an optional part that can be repeated an arbitrary number of times.
  \item \texttt{[...]\^{}\{1..n\}} indicates an part that can be repeated an arbitrary number of times, at least once.
  \item \texttt{[...,]\^{}\{0/1..n\}} indicates a part that can be repeated an arbitrary number of times, each repetition being separated by a \texttt{,} (\emph{but there is actually no \texttt{,} after the last repetition}).
\end{itemize}

\subsubsection{UML}

\begin{figure}[!h]
  \centering
  \includegraphics[width=\linewidth]{UML}
  \caption{UML format used.}
  \label{fig:UML}
\end{figure}

