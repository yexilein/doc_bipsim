% ============================================
%  Article Class (This is a LaTeX2e document)
% ============================================
\documentclass[12pt]{scrartcl}
\usepackage[english]{babel}
\usepackage{enumitem}
\usepackage[round]{natbib}
\usepackage{color}
\usepackage{array}

\newcommand\reft[3][]{#2~\ref{#3}#1}
\newcommand\refp[3][]{(#2~\ref{#3}#1)}
\newcommand\refsect[1]{\reft{Section}{#1}}
\newcommand\refsecp[1]{\refp{Sec.}{#1}}
\newcommand\reftabt[1]{\reft{Table}{#1}}
\newcommand\reftabp[1]{\refp{Tab.}{#1}}

% ============
%  Algorithms
% ============
\usepackage{algorithm2e}
\SetKwProg{Fn}{Function}{}{}
\newcommand\refalgt[1]{\reft{Algorithm}{#1}}
\newcommand\refalgp[1]{\refp{Alg.}{#1}}

% ======
%  Math
% ======
\usepackage{amsmath}
\usepackage{amsthm}
\newtheorem{thm}{Theorem}[section]
\newtheorem{cor}[thm]{Corollary}
\newtheorem{lem}[thm]{Lemma}
\newtheorem{prop}[thm]{Proposition}
\newtheorem{property}[thm]{Property}
\theoremstyle{definition}
\newtheorem{defn}[thm]{Definition}
\newtheorem{assum}[thm]{Assumption}
\theoremstyle{remark}
\newtheorem{rem}[thm]{Remark}
\numberwithin{equation}{section}
\usepackage{amssymb}
\newcommand{\prob}[1]{\mathbb{P}\left(#1\right)}

% ====================
%  Chemical reactions
% ====================
\usepackage[version=3]{mhchem}
\newcommand{\reactionRev}[4]{#1 \ce{<=>[#3][#4]} #2}
\newcommand{\reactionIrr}[4]{#1 \ce{->[#3][#4]} #2}

% ============================
%  Figures and relative paths
% ============================
\usepackage{graphicx}
\graphicspath{{figures/}{cati/figures/}}
\usepackage{import}
\makeatletter
  \def\relativepath{\import@path}
\makeatother
\newcommand\reffigt[2][]{\reft[#1]{Figure}{#2}}
\newcommand\reffigp[2][]{\refp[#1]{Fig.}{#2}}


% ==========
%  Document
% ==========
\begin{document}

\title{BiPSim: a flexible and generic stochastic simulator for cell processes - Supplementary Information}%
\author{}%
%\address{er}%
%\thanks{zq}%
\date{\today}%
\maketitle

\newpage

\tableofcontents

\newpage

\section{Introduction}

This document walks through the choices in design that we made while developing BiPSim
(the code is documented using Doxygen, technical details are therefore best found in the Doxygen-generated manual).
It highlights the central classes in BiPSim's architecture and how classes interact.

The first section focuses on the base components of the simulator, reactants and reactions.
It starts with a global overview of all rectants and reactants.
A second subsection describes the same element again,
but goes further into hypotheses and critical design elements.
Appendices are added to describe elements that have been important in the simulator development
but did not fit naturally in the main document (testing strategies, utility classes, etc.).

The second section focuses on the implementation of Gillespie's Stochastic Simulation Algorithm (SSA).
It starts with an overview of the SSA, insisting the trade-off between two of its
components: selecting the next reaction to perform and updating reaction rates.
The second subsection presents various strategies to select the next reaction implemented in BiPSim:
the direct method, the binary search and the composition-rejection method.
The third subsection presents efficient strategies to update reaction rates as implemented
in BiPSim, which play an essential role in the final performance of the simulator.
Appendices provide additional information about the implementation and some perspectives.

\section{Implementation of reactions and reactants}

\subimport{global/}{global.tex}
\subimport{detailed/}{detailed.tex}
\clearpage

\subimport{appendix/}{appendix}
\clearpage

\section{Implementation of Gillespie's Stochatic Simulation Algorithm}

\subimport{cati/}{cati.tex}
\clearpage

\bibliographystyle{plainnat}
\bibliography{simulatorDesign}

\end{document}
% ----------------------------------------------------------------
