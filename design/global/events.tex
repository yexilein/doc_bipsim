\subsubsection{Events}

\texttt{Event}s enable users to change molecule numbers outside of the solver loop at specific times~\reffigp{fig:events}. A \texttt{Simulation} instance handles both a \texttt{Solver} instance and an \texttt{EventHandler} instance. Every time an event timing is reached, the solver loop is stopped, the event(s) is (are) performed, the solver is reinitialized and the simulation resumes. Different \texttt{Event} implementations are offered to modify molecule numbers in a convenient way.

\begin{figure}[!h]
  \centering
  \includegraphics[width=\linewidth]{events}
  \caption{\texttt{Event}s: another way to modify chemical concentrations aside from reactions, \textit{e.g.} to simulate the injection of a chemical inside a cell.}
  \label{fig:events}
\end{figure}
