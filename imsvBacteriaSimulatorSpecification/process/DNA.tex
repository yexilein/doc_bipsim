\subsubsection{DNA replication}
See illustration how the cell chromosome is used. Some more information may be needed TBD.
When a chromosome is fully duplicated (when the first column of the cell chromosome only contains -2), the process consists of deleting the current chromosome and creating 2 new ones with the correct initialization. TBD do we clean the chromosomes? Typically, when the chromosome was manipulated.

Needs also other things but from the DNA point of view, there seems to have enough information.

\subsubsection{DNA movement}
\textcolor[rgb]{1.00,0.00,0.00}{What do we do when a gene change of volume due to condensation or segregation for example. Normally, a matter of changing the number of the volume in the cell chromosome and the corresponding volume chromosomes. But assume that anything that is currently binded to the DNA is the property of the DNA and was 'erased' from everywhere else. Also assume that volume chromosome only contains the strictly necessary information about the DNA in the volume and that it is a state with changing size.}

\subsubsection{DNA manipulations}
\paragraph{Codon aggregation damage}
\textcolor[rgb]{1.00,0.00,0.00}{Gap site, Abasic site, Sugar-phosphate, Base, Intrastrand cross link, Strand break, Holliday junction}

\paragraph{Codon aggregation insertion}
Insertion of one (or several) lines in the DNA states.

\paragraph{Codon aggregation deletion}
Putting 0 in the corresponding places and not deletion of the line(s) because a codon aggregation can be deleted in a fork.

\paragraph{Codon aggregation repair}
\textcolor[rgb]{1.00,0.00,0.00}{How?}


\subsubsection{DNA compaction}
\textcolor[rgb]{1.00,0.00,0.00}{Condensation, ”clamping” of the DNA by structural maintenance of chromosome (SMC) proteins, supercoiling, macromolecular crowding, charge neuralization?}
If spatialization is used, condensation and segragation might be modelled directly. Supercoiling needs another state.
\textcolor[rgb]{1.00,0.00,0.00}{Compactation should also impact the accessibility of the chromosome.}

\subsubsection{DNA segregation}
\textcolor[rgb]{1.00,0.00,0.00}{How?}
