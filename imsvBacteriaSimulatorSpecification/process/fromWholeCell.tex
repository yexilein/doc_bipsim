Police barr\'{e}e = un process de wholeCell que j'ai remis en haut, sans pour autant l'avoir mod\'{e}liser ou avoir les informations n\'{e}cessaires dans les \'{e}tats.
\begin{itemize}
  \item \sout{Chromosome condensation: DNA clamping by SMC complexes}
  \item \sout{Chromosome segregation}
  \item \sout{Cytokinesis: pinching of the cell membrane}
  \item \sout{DNA damage: Gap site, Abasic site, Sugar-phosphate, Base, Intrastrand cross link, Strand break, Holliday junction}
  \item \sout{DNA repair}
  \item \sout{DNA supercoiling}
  \item FtsZ polymerization
  \item Host interaction: des trucs mais en quoi c'est utile ?
  \item Macromolecular complexation: models the formation of all macromolecular complexes except the 30S and 50S ribosomal particles, the 70S ribosome, the FtsZ ring, and the oriC DnaA complex
  \item \sout{Metabolism: models the import of extracellular nutrients and their conversion into macromolecule building blocks}
  \item Protein activation:  implements a Boolean model of their effects on the functional state – enzymatically competent or incompetent – of mature proteins
  \item Protein decay: models the degradation of protein monomers, macromolecular complexes, cleaved signal sequences, and prematurely aborted polypeptides as well as the misfolding and refolding of protein monomers and complexes
  \item \sout{Protein folding}
  \item \sout{Protein modification: models protein covalent modification including phosphorylation, lipoyl transfer, and α-glutamate ligation}
  \item \sout{Protein processing I: models N-terminal formylmethionine deformylation and N-terminal methionine cleavage, the first steps in post-translational processing. What's that???}
  \item \sout{Protein processing II: models the third step of post-translational processing: lipoprotein diacylglyceryl adduction and lipoprotein and secreted protein signal peptide cleavage. What's that?}
  \item \sout{Protein translocation: models membrane and extracellular protein localization, the second step in post-translational processing}
  \item \sout{Replication}
  \item Replication initialization: determines when during the cell cycle chromosome duplication begins. Uses the protein DnaA (MG469)
  \item \sout{Ribosome assembly:  models the enzyme-catalyzed formation of 30S and 50S ribosomal particles}
  \item \sout{RNA decay: decays all species of RNA, and at all maturation states including aminoacylated states}
  \item \sout{RNA modification: the exact role of rRNA modification is unknown. This process models tRNA and rRNA modification}
  \item \sout{RNA processing:  models operonic RNA cleavage into individual RNA gene products. Something about operons.}
  \item Terminal organelle assembly: models the assembly of the protein content of the terminal organelle
  \item \sout{Transcription: For simplicity, our model doesn’t represent this phenomenon, allowing translation only of completed mRNAs}
  \item Transcription regulation: models the binding of transcriptional regulators to promoters and the fold-change effect of transcriptional regulators on the affinity of RNA polymerase for individual promoters.
  \item \sout{Translation}
  \item \sout{tRNA aminocylation}
\end{itemize}