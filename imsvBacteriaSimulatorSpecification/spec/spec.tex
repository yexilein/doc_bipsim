\begin{description}
  \item[Fxx] The simulator shall be able to simulate the following scenario:
  \begin{itemize}
    \item different kind of (constant) extracellular conditions;
    \item upshift and downshift;
    \item entry in stationary phase.
  \end{itemize}
  \item[Fxx] It shall be possible to ‘kill’ a cell among others.
  \item[Fxx] It shall be possible to initialize the state of the simulation. (A nice to have feature: moreover some ‘aggregation’ or ‘division’ of the state may be needed for such an initialization. For instance, if a cell has been simulated with one volume, a state ‘division’ is needed to initialize a simulation with a cell simulated on several volumes.)
  \item[Fxx] Several stopping criterion shall be implemented:
  \begin{itemize}
    \item at the end of cellular division;
    \item simulated time exceeds a certain value;
    \item simulation time exceeds a certain value.
  \end{itemize}
      Any combination of criteria may be active.
  \item[Fxx] Typical values to be simulated are:
  \begin{itemize}
    \item 3e6 proteins (including ribosomes);
    \item 1e3 mRNA;
    \item 5e3 genes, each gene corresponding to 200 codons;
    \item 1h duration.
  \end{itemize}
  \item[Fxx] A simulation of the lifetime (cell reproduction) of a cell (Bacillus Subtilis) shall last 8 hours at most.
  \\

  \item[Fxx] The simulator shall output an ASCII logfile containing at least:
  \begin{itemize}
    \item the operating system;
    \item the version of the simulator and softs used;
    \item the date, the computer and the user;
    \item the localization of the data and results files.
  \end{itemize}
  \item[Fxx] The simulator shall be composed of 3 modules:
  \begin{itemize}
    \item the first module outputs the description of the simulation from queries of the database. This description is contained in an ASCII file that uses SBML tags to the maximum;
    \item the second module is the simulation itself. From the ASCII file provided by the first module, it outputs the simulation results in a set of ASCII files. These files shall be readable from Excel 2013 or Matlab 2012a;
    \item the third module is the analysis and visualization of the simulation results provided by the second module. (A nice to have feature is a visual representation of the simulation.)
  \end{itemize}
  \item[Fxx] The functionalities of the simulator shall be separated from its data and results.
  \item[Fxx] It shall be possible to launch a simulation interactively or by batch.
  \item[Fxx] It shall be possible to select the data to be recorded. By default all data shall be recorded.
  \\

  \item[Fxx] The cellular processes described in WholeCell shall be simulated. In addition, the stringent response (via RelA) shall be simulated.
  \item[Fxx] The simulator shall be able to simulate different kind of process modeling. Typically, 4 kinds of modelling are envisaged for the process S -> P
  \begin{itemize}
    \item P = f(S): this is a static (steady state) relation. Typically used when the duration is really fast;
    \item dot{P} = f(S): this is an average kinematic relation. It is what is typically obtained by dynamic RBA;
    \item P(nk+1) = f(P(nk),...) = P(nk)+1: this is a quantification of the number of products. No averaging, we are at the level of one bacteria. The event (time) is deterministic;
    \item P(t+) = f(P(t-),...) = P(t-)+1: with stochastic behavior of the event.
  \end{itemize}
      This process modeling can all be present for a description of a bacteria however a particular process shall be modeled only by one of this modeling.
  \item[Fxx] This specification assumes that it is possible to obtain metabolite pool during the simulations. The interface between the different modeling granularities is the pool of metabolites.
  \item[Fxx] It shall be possible to impose the output of a particular process, typically for debug and validation.
  \\
  
  \item[Dxx] The development of the simulator shall be versioned.
  \item[Dxx] The simulator shall be documented with:
  \begin{itemize}
    \item a user manual;
    \item a developer manual.
  \end{itemize}
  \item[Dxx] The simulator shall be validated on Bacillus Subtilis.
\end{description}


