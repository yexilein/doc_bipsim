% ----------------------------------------------------------------
% Article Class (This is a LaTeX2e document)  ********************
% ----------------------------------------------------------------
\documentclass[12pt]{article}
\usepackage[english]{babel}
\usepackage{amsmath,amsthm}
\usepackage{amsfonts}
\usepackage{multirow}
\usepackage{color,ulem}
\usepackage{graphicx}
%\usepackage{mdframed}
\usepackage[framemethod=TikZ]{mdframed}
\mdfdefinestyle{MyFrame}{%
    linecolor=blue,
    outerlinewidth=1pt,
%    roundcorner=20pt,
%    innertopmargin=\baselineskip,
    innertopmargin=3pt,
%    innerbottommargin=\baselineskip,
%    innerrightmargin=20pt,
%    innerleftmargin=20pt,
%    backgroundcolor=gray!50!white
    }
% THEOREMS -------------------------------------------------------
\newtheorem{thm}{Theorem}[section]
\newtheorem{cor}[thm]{Corollary}
\newtheorem{lem}[thm]{Lemma}
\newtheorem{prop}[thm]{Proposition}
\theoremstyle{definition}
\newtheorem{defn}[thm]{Definition}
\newtheorem{assum}[thm]{Assumption}
\theoremstyle{remark}
\newtheorem{rem}[thm]{Remark}
\numberwithin{equation}{section}

\usepackage{mhchem}
\newcommand{\reactionRev}[4]{#1 \ce{<=>[#3][#4]} #2}
\newcommand{\reactionIrr}[4]{#1 \ce{->[#3][#4]} #2}

% ----------------------------------------------------------------
\begin{document}
\normalem

\title{{IMSV}: bacteria simulator specifications and architecture}%
\author{M.~Dinh}%
%\address{er}%
%\thanks{zq}%
\date{}%
% ----------------------------------------------------------------
\maketitle
\begin{abstract}
  XXX
\end{abstract}
% ----------------------------------------------------------------
\section{Introduction}

The objective of the project is to develop a simulator for a generic bacteria. The project is performed in parallel to the creation of an ontology and the associated database that describes living systems. This document presents the specifications of the simulator.


\section{Specifications}
\begin{description}
  \item[Fxx] The simulator shall be able to simulate the following scenario:
  \begin{itemize}
    \item different kind of (constant) extracellular conditions;
    \item upshift and downshift;
    \item entry in stationary phase.
  \end{itemize}
  \item[Fxx] It shall be possible to ‘kill’ a cell among others.
  \item[Fxx] It shall be possible to initialize the state of the simulation. (A nice to have feature: moreover some ‘aggregation’ or ‘division’ of the state may be needed for such an initialization. For instance, if a cell has been simulated with one volume, a state ‘division’ is needed to initialize a simulation with a cell simulated on several volumes.)
  \item[Fxx] Several stopping criterion shall be implemented:
  \begin{itemize}
    \item at the end of cellular division;
    \item simulated time exceeds a certain value;
    \item simulation time exceeds a certain value.
  \end{itemize}
      Any combination of criteria may be active.
  \item[Fxx] Typical values to be simulated are:
  \begin{itemize}
    \item 3e6 proteins (including ribosomes);
    \item 1e3 mRNA;
    \item 5e3 genes, each gene corresponding to 200 codons;
    \item 1h duration.
  \end{itemize}
  \item[Fxx] A simulation of the lifetime (cell reproduction) of a cell (Bacillus Subtilis) shall last 8 hours at most.
  \\

  \item[Fxx] The simulator shall output an ASCII logfile containing at least:
  \begin{itemize}
    \item the operating system;
    \item the version of the simulator and softs used;
    \item the date, the computer and the user;
    \item the localization of the data and results files.
  \end{itemize}
  \item[Fxx] The simulator shall be composed of 3 modules:
  \begin{itemize}
    \item the first module outputs the description of the simulation from queries of the database. This description is contained in an ASCII file that uses SBML tags to the maximum;
    \item the second module is the simulation itself. From the ASCII file provided by the first module, it outputs the simulation results in a set of ASCII files. These files shall be readable from Excel 2013 or Matlab 2012a;
    \item the third module is the analysis and visualization of the simulation results provided by the second module. (A nice to have feature is a visual representation of the simulation.)
  \end{itemize}
  \item[Fxx] The functionalities of the simulator shall be separated from its data and results.
  \item[Fxx] It shall be possible to launch a simulation interactively or by batch.
  \item[Fxx] It shall be possible to select the data to be recorded. By default all data shall be recorded.
  \\

  \item[Fxx] The cellular processes described in WholeCell shall be simulated. In addition, the stringent response (via RelA) shall be simulated.
  \item[Fxx] The simulator shall be able to simulate different kind of process modeling. Typically, 4 kinds of modelling are envisaged for the process S -> P
  \begin{itemize}
    \item P = f(S): this is a static (steady state) relation. Typically used when the duration is really fast;
    \item dot{P} = f(S): this is an average kinematic relation. It is what is typically obtained by dynamic RBA;
    \item P(nk+1) = f(P(nk),...) = P(nk)+1: this is a quantification of the number of products. No averaging, we are at the level of one bacteria. The event (time) is deterministic;
    \item P(t+) = f(P(t-),...) = P(t-)+1: with stochastic behavior of the event.
  \end{itemize}
      This process modeling can all be present for a description of a bacteria however a particular process shall be modeled only by one of this modeling.
  \item[Fxx] This specification assumes that it is possible to obtain metabolite pool during the simulations. The interface between the different modeling granularities is the pool of metabolites.
  \item[Fxx] It shall be possible to impose the output of a particular process, typically for debug and validation.
  \\
  
  \item[Dxx] The development of the simulator shall be versioned.
  \item[Dxx] The simulator shall be documented with:
  \begin{itemize}
    \item a user manual;
    \item a developer manual.
  \end{itemize}
  \item[Dxx] The simulator shall be validated on Bacillus Subtilis.
\end{description}




\section{Data and states}
It is recommended to implement the simulator with the following architecture
\begin{description}
  \item Simulation object
  \begin{enumerate}
    \item Time (state)
    \item Interface between extracellular conditions (processes)
    \item Extracellular conditions 1 (object):
    \begin{enumerate}
      \item concentration of entities (glucose, metabolites, AA… or whatever) (state)
      \item evolution of the concentration (process)
    \end{enumerate}
    \item Extracellular conditions 2
    \item ...
    \item Extracellular conditions n
    \item Cell 1 (object)
    \begin{enumerate}
      \item Interface between volumes (which volume interact with which other volume and the corresponding surface) (object)
      \item Volume 1 (object)
      \begin{enumerate}
        \item Extracellular conditions (processes, numbers, surfaces) (should allow for interaction between cells through a virtual extracellular conditions)
        \item Volume (state)
        \item Number of ribosome in different states or specific ribosome and its state (Ribosome i, free occupied…) (state)
        \item Number of mRNA in different states or specific mRNA and its state (state)
        \item ...
        \item Number of metabolites in different states (state)
        \item Volume Chromosome 1: table of the same size describing the state of the chromosome (state)
        \item Volume Chromosome 2
        \item ...
        \item Volume Chromosome n (the same number as for the cell) (state)
        \item Internal (to the volume) processes
      \end{enumerate}
      \item Volume 2
      \item ...
      \item Volume n
      \item Cell Chromosome 1 (object):
      \begin{enumerate}
        \item Table of ‘mère’ / ‘filles’ (state)
        \item Table (size = number of structure (gene or codon or groups of codons) x maximal number of ‘brindilles’) with the number of the cell containing the structure. By convention, 0 = non active/does not exist; -1 = to fork (go the chromosomes ‘filles’); -2 = from fork (go the chromosome ‘mère’) (state)
        \item Description of the structures (which AA needed), description of the genes (which structures, to produce what) (data)
      \end{enumerate}
      \item Cell Chromosome 2 if needed
      \item ...
      \item Cell Chromosome n if needed
      \item Internal (to the cell) processes, typically replication or DNA movement from a volume to another one
      \end{enumerate}
    \item Cell 2
    \item ...
    \item Cell n
  \end{enumerate}
\end{description}
Basically, cell only communicate with extracellular conditions. Each cell is divided into volumes that can communicate either with extracellular conditions or other volumes of the same cell. The chromosome is divided into 'cell chromosome' which handles the continuity of the chromosome and 'volume chromosome' which handles transcription and translation.



\subsection{Chromosome}
There are 'cell chromosome' and 'volume chromosome' because we wish that only a higher level object may access to the data and state of a lower object; this distinction is otherwise not needed and the chromosome object should be placed at cell level. 'Cell chromosome' mainly handles the continuity of the chromosome of the cell through all the volumes. 'Volume chromosome' directly handles the biological process internally to each volume. 'Cell chromosome' and 'volume chromosome' must be coherent.


\paragraph{Representation} A chromosome is represented in a condensed way in the simulator. In fact depending on the aggregation performed, the representation can be more or less fine.
\begin{center}
\begin{tabular}{|c|c|}
  \hline
  % after \\: \hline or \cline{col1-col2} \cline{col3-col4} ...
  NTP 1  & \multirow{1}{*}{NTP aggregation 1} \\
  \hline
  NTP 2  & \multirow{3}{*}{NTP aggregation 2} \\
  NTP 3  &  \\
  NTP 4  &  \\
  \hline
  NTP 5  & \multirow{5}{*}{NTP aggregation 3} \\
  NTP 6  &  \\
  NTP 7  &  \\
  NTP 8  &  \\
  NTP 9  &  \\
  \hline
  NTP 10 & \multirow{2}{*}{NTP aggregation 4} \\
  NTP 11 &  \\
  \hline
  NTP 12 & \multirow{10}{*}{NTP aggregation 5} \\
  NTP 13 &  \\
  NTP 14 &  \\
  NTP 15 &  \\
  NTP 16 &  \\
  NTP 17 &  \\
  NTP 18 &  \\
  NTP 19 &  \\
  NTP 20 &  \\
  NTP 21 &  \\
  \hline
  NTP 22 & \multirow{9}{*}{NTP aggregation 6} \\
  NTP 23 &  \\
  NTP 24 &  \\
  NTP 25 &  \\
  NTP 26 &  \\
  NTP 27 &  \\
  NTP 28 &  \\
  NTP 29 &  \\
  \hline
  NTP 30 & \multirow{1}{*}{NTP aggregation 7} \\
  \hline
\end{tabular}
\end{center}
Note that with this aggregation, there are some conflicts between DNA replication / transcription and DNA damage / repair / mutation... Unless some rules are enforced to solve the conflicts, it is not possible to have different kind of process be performed the same time on a NTP aggregation.
\begin{assum}
\begin{enumerate}
  \item To avoid change of a volume state to another one due to, for instance, a polymerase changing of volume, it is assumed that a gene is necessary contained in one volume;
  \item The decomposition into NTP aggregations applies to any biological process related to DNA (replication, transcription, translation and manipulation).
\end{enumerate}
\end{assum}
\noindent This assumption should not too drastic as a gene is much smaller than a chromosome.


\subsubsection{Cell Chromosome}
\paragraph{Evolution state} For DNA replication purpose, it is needed to represent fork. It is assumed that the maximal number of forks $n$ in known and that the considered chromosome is represented by $m$ NTP aggregations. Then, the chromosome can be represented by 3 tables:
\begin{enumerate}
  \item a main table of size $m \times \sum_{i=0}^n2^i$ containing the number of the volume where it is. Moreover, -1 means go to 'mother', -2 means go to 'daughters' and -3 means go both to 'mother' and 'daughters';
  \item a 'mother' table of size $1 \times \sum_{i=0}^n2^i$ representing one strand of the fork;
  \item a 'daughters' table of size $2 \times \sum_{i=0}^n2^i$ representing the two last strands of the fork.
\end{enumerate}

\medskip

We now illustrate how it works. Assume that the chromosome (with $n=2$ and $m=10$) does not have any fork yet and is contained in the volume 2, then the tables are
\begin{center}
  \begin{tabular}{|c|r|r|r|r|r|r|r|}
     \hline
     % after \\: \hline or \cline{col1-col2} \cline{col3-col4} ...
     \multirow{10}{*}{Main table} & 2 & 0 & 0 & 0 & 0 & 0 & 0 \\
     & 2 & 0 & 0 & 0 & 0 & 0 & 0 \\
     & 2 & 0 & 0 & 0 & 0 & 0 & 0 \\
     & 2 & 0 & 0 & 0 & 0 & 0 & 0 \\
     & 2 & 0 & 0 & 0 & 0 & 0 & 0 \\
     & 2 & 0 & 0 & 0 & 0 & 0 & 0 \\
     & 2 & 0 & 0 & 0 & 0 & 0 & 0 \\
     & 2 & 0 & 0 & 0 & 0 & 0 & 0 \\
     & 2 & 0 & 0 & 0 & 0 & 0 & 0 \\
     & 2 & 0 & 0 & 0 & 0 & 0 & 0 \\
     \hline \hline
     'Mother' table & 0 & 0 & 0 & 0 & 0 & 0 & 0 \\
     \hline \hline
     \multirow{2}{*}{'Daughters' table} & 0 & 0 & 0 & 0 & 0 & 0 & 0 \\
     & 0 & 0 & 0 & 0 & 0 & 0 & 0 \\
     \hline
   \end{tabular}
\end{center}
Now assume that the NTP aggregation 1 to 5 and 7 to 10 were replicated, one strand being in the volume 2 and the other one in the volume 1, the tables are
\begin{center}
  \begin{tabular}{|c|r|r|r|r|r|r|r|}
     \hline
     % after \\: \hline or \cline{col1-col2} \cline{col3-col4} ...
     \multirow{10}{*}{Main table} & 0 &  2 &  1 & 0 & 0 & 0 & 0 \\
     &  0 &  2 &  1 & 0 & 0 & 0 & 0 \\
     &  0 &  2 &  1 & 0 & 0 & 0 & 0 \\
     &  0 &  2 &  1 & 0 & 0 & 0 & 0 \\
     & -2 &  2 &  1 & 0 & 0 & 0 & 0 \\
     &  2 & -1 & -1 & 0 & 0 & 0 & 0 \\
     & -2 &  2 &  1 & 0 & 0 & 0 & 0 \\
     &  0 &  2 &  1 & 0 & 0 & 0 & 0 \\
     &  0 &  2 &  1 & 0 & 0 & 0 & 0 \\
     &  0 &  2 &  1 & 0 & 0 & 0 & 0 \\
     \hline \hline
     'Mother' table & 0 & 1 & 1 & 0 & 0 & 0 & 0 \\
     \hline \hline
     \multirow{2}{*}{'Daughters' table} & 2 & 0 & 0 & 0 & 0 & 0 & 0 \\
     & 3 & 0 & 0 & 0 & 0 & 0 & 0 \\
     \hline
   \end{tabular}
\end{center}
Assume that there is another fork in the volume 1 for the NTP aggregation 1 only contained in volume 1 also while a part of the strand moved to volume 3, then the tables are
\begin{center}
  \begin{tabular}{|c|r|r|r|r|r|r|r|}
     \hline
     % after \\: \hline or \cline{col1-col2} \cline{col3-col4} ...
     \multirow{10}{*}{Main table} & 0 &  2 & -2 &  1 &  1 & 0 & 0 \\
     &  0 &  2 &  1 & -1 & -1 & 0 & 0 \\
     &  0 &  2 &  1 &  0 &  0 & 0 & 0 \\
     &  0 &  2 &  1 &  0 &  0 & 0 & 0 \\
     & -2 &  2 &  1 &  0 &  0 & 0 & 0 \\
     &  2 & -1 & -1 &  0 &  0 & 0 & 0 \\
     & -2 &  2 &  3 &  0 &  0 & 0 & 0 \\
     &  0 &  2 &  3 &  0 &  0 & 0 & 0 \\
     &  0 &  2 &  3 &  0 &  0 & 0 & 0 \\
     &  0 &  2 &  3 & -1 & -1 & 0 & 0 \\
     \hline \hline
     'Mother' table & 0 & 1 & 1 & 3 & 3 & 0 & 0 \\
     \hline \hline
     \multirow{2}{*}{'Daughters' table} & 2 & 0 & 4 & 0 & 0 & 0 & 0 \\
     & 3 & 0 & 5 & 0 & 0 & 0 & 0 \\
     \hline
   \end{tabular}
\end{center}
Finally assume that the last fork is performed on NTP aggregations 1 to 5 (even if it is a weird as it would mean that the replication is performed only in one direction, see the continuation of this example below) and that the resulting strands are contained in volume 3 even if the 'mother' strand is contained in volume 2, then the tables are
\begin{center}
  \begin{tabular}{|c|r|r|r|r|r|r|r|}
     \hline
     % after \\: \hline or \cline{col1-col2} \cline{col3-col4} ...
     \multirow{10}{*}{Main table} & 0 & -2 & -2 &  1 &  1 & 3 & 3 \\
     &  0 &  0 &  1 & -1 & -1 &  3 &  3 \\
     &  0 &  0 &  1 &  0 &  0 &  3 &  3 \\
     &  0 &  0 &  1 &  0 &  0 &  3 &  3 \\
     & -2 & -2 &  1 &  0 &  0 &  3 &  3 \\
     &  2 & -1 & -1 &  0 &  0 & -1 & -1 \\
     & -2 &  2 &  3 &  0 &  0 &  0 &  0 \\
     &  0 &  2 &  3 &  0 &  0 &  0 &  0 \\
     &  0 &  2 &  3 &  0 &  0 &  0 &  0 \\
     &  0 &  2 &  3 & -1 & -1 & -1 & -1 \\
     \hline \hline
     'Mother' table & 0 & 1 & 1 & 3 & 3 & 2 & 2 \\
     \hline \hline
     \multirow{2}{*}{'Daughters' table} & 2 & 6 & 4 & 0 & 0 & 0 & 0 \\
     & 3 & 7 & 5 & 0 & 0 & 0 & 0 \\
     \hline
   \end{tabular}
\end{center}


\paragraph{Original chromosome data} This state contains other needed information for transcription and translation purposes. Moreover, they do not change as often that the evolution state but it can be the case whenever something happens to the DNA. The needed information are
\begin{itemize}
  \item Transcription: the location of the genes, which bases are needed, \textcolor[rgb]{1.00,0.00,0.00}{kinetic parameters (typically how long does it takes to replicate or transcript it)};
  \item Translation: the needed AA, \textcolor[rgb]{1.00,0.00,0.00}{the protein produced which is the property of the gene};
  \item \textcolor[rgb]{1.00,0.00,0.00}{Manipulation: is it an original NTP aggregation?}
\end{itemize}

\medskip

With a 10 NTP aggregations chromosome, the following original information state table
\begin{center}
\footnotesize
  \begin{tabular}{|ccccc|c|cccc|ccccc|}
    \hline
    % after \\: \hline or \cline{col1-col2} \cline{col3-col4} ...
    \multicolumn{5}{|c|}{Gene} &  & \multicolumn{4}{|c|}{Base} & \multicolumn{5}{|c|}{AA} \\
    number & oriC & terC & Start & Stop & Nb of NTPs & A & T & C & G & AA 1 & ... & AA i & ... & AA n \\
    \hline
     1  &   1  &   0  &  0    &   0  & 1 &    1   &    1   &    0   &    1   &   0  & ... &  0   & ... &  0 \\
     2  &   0  &   0  &  1    &   0  & 1 &    1   &    0   &    1   &    1   &   0  & ... &  0   & ... &  0 \\
     2  &   0  &   0  &  0    &   0  & 5 &    5   &    4   &    5   &    1   &   1  & ... &  2   & ... &  2 \\
     2  &   0  &   0  &  0    &   0  & 4 &    2   &    1   &    5   &    4   &   1  & ... &  1   & ... &  0 \\
     2  &   0  &   0  &  0    &   1  & 1 &    2   &    1   &    0   &    0   &   0  & ... &  0   & ... &  0 \\
     3  &   0  &   1  &  0    &   0  & 1 &    0   &    1   &    1   &    1   &   0  & ... &  0   & ... &  0 \\
     4  &   0  &   0  &  1    &   0  & 1 &    1   &    0   &    1   &    1   &   0  & ... &  0   & ... &  0 \\
     4  &   0  &   0  &  0    &   0  & 5 &    1   &    1   &   12   &    1   &   0  & ... &  0   & ... &  1 \\
     4  &   0  &   0  &  0    &   0  & 3 &    3   &    3   &    0   &    3   &   0  & ... &  3   & ... &  0 \\
     4  &   0  &   0  &  0    &   1  & 1 &    2   &    1   &    0   &    0   &   0  & ... &  0   & ... &  0 \\
    \hline
  \end{tabular}
\end{center}
indicates
\begin{itemize}
  \item oriC as gene 1, encoded in NTP aggregation 1. The corresponding codon is any combination of bases ATG;
  \item terC as gene 3, encoded in NTP aggregation 6. The corresponding codon is any combination of bases TCG;
  \item 2 real genes (2 and 4) encoded between NTP aggregations 2 and 5 and between NTP aggregations 7 and 10;
  \item both genes have the same start and stop codon (coded by any combination of ACG and AAT) and the do not need any AA;
  \item the translation elongation of NTP aggregation 3 would need 4 AA 1, ..., 1 AA i, ... and 8 AAn. Its transcription needs the equivalent of 5 bases A, 4 bases T, 5 bases C and 1 base G whereas its replication would need the double of bases;
  \item in my little head, a codon is composed of 3 bases and each codon encodes one AA. The column 'Number of codons' should be consistent with the 'Base' and 'AA' columns. The AA coded bu aggregation 3 are all displayed in the table whereas it is not the case for codon aggregation 4 (it lacks 2 AA).
\end{itemize}
Note that with this kind of modeling, the oriC, terC, start and stop codons can be aggregated with others. \emph{This is not recommended however}. Another way would be to delete the oriC and terC property and reserve some values (1 and -1 for example) in the Gene number to them.

\textcolor[rgb]{1.00,0.00,0.00}{Note, for gene 4 and more generally in this part of the chromosome, the replication and the translation go in reverse way => conflict.}

\medskip

\textcolor[rgb]{1.00,0.00,0.00}{In fact, not gene but rather operons are described for transcription. Delete the part from AA on the chromosome description and add something about the the operons. Dissociate for each operons the number of mRNAs possible that are created from on operon, and for each mRNA the number for proteins that are created.}

\medskip

\textcolor[rgb]{1.00,0.00,0.00}{\emph{In fact, it might be easier to not use the codon aggregations and directly code every codon...}}

\paragraph{Current chomosome state}
\textcolor[rgb]{1.00,0.00,0.00}{It is the same as the original data except that it indicates the current state of the DNA. The original table allows to retrieve the original DNA when a DNA repair has been performed on a damaged DNA. In that sense, the current table also contains infirmation about the DNA damages.}


\paragraph{Gene state}
\textcolor[rgb]{1.00,0.00,0.00}{Should just be something that tells for each gene what it produces. As I see only proteins at the moment as output, is there something else potentially? How do we store this information?}



\subsubsection{Volume Chromosome}
Volume chromosome contains information about the gene and codon aggregations that are inside the volume. They also contains information necessary for an independent processing of transcription and translation, but not DNA replication (which is a cell process).

\medskip

The information needed for modeling are
\begin{itemize}
  \item what are the different state, more precisely, how other entities can bind to DNA for transcription;
  \item in what step is the elongation (all the bases are in place?);
  \item in order to manage conflict with replication, a state about occupation by replication is added.
\end{itemize}
The volume chromosome is organized by operons with the same codon aggregations. Also adds an information about the tracking of the strand of the operon.

\textcolor[rgb]{1.00,0.00,0.00}{Mind backtrack: per operon, the number, the strand it comes from.}


\subsection{mRNA}
A mRNA has the same kind of structure as an operon. \textcolor[rgb]{1.00,0.00,0.00}{Add binding site (and a stop site?) to Chromosome characteristics.}
\begin{center}
  \tiny
  \begin{tabular}{|c|c|c|c|c||c|c|c|c|c|c|c|}
    \hline
    % after \\: \hline or \cline{col1-col2} \cline{col3-col4} ...
    Codon & AA 1 & AA 2 & AA 3 & AA 4 & Aggregation & Binding site & Stop site & AA 1 & AA 2 & AA 3 & AA 4 \\
    \hline
      1   &  0   &  0   &   0  &   0  &       1     &      1       &     0     &   0  &  0   &   0  &   0  \\
    \hline
      2   &  0   &  0   &   1  &   0  &  \multirow{3}{*}{2} & \multirow{3}{*}{0} & \multirow{3}{*}{0} & \multirow{3}{*}{1} & \multirow{3}{*}{0} & \multirow{3}{*}{2} & \multirow{3}{*}{0} \\
      3   &  1   &  0   &   0  &   0  & & & & & & & \\
      4   &  0   &  0   &   1  &   0  & & & & & & & \\
    \hline
      5   &  0   &  0   &   0  &   1  &  \multirow{2}{*}{3} & \multirow{2}{*}{0} & \multirow{2}{*}{0} & \multirow{2}{*}{1} & \multirow{2}{*}{0} & \multirow{2}{*}{0} & \multirow{2}{*}{1} \\
      6   &  1   &  0   &   0  &   0  & & & & & & & \\
    \hline
      7   &  0   &  0   &   0  &   0  &      4      &      0       &     1     &   0  &  0   &   0  &   0  \\
    \hline
  \end{tabular}
\end{center}
So an mRNA is represented by
\begin{enumerate}
  \item what it produces (how this information is stored?);
  \item the aggregated table of information;
  \item per aggregation, the state. Basically, for a brief description, we assume that the translation is performed in the following steps
  \begin{enumerate}
    \item binding with 30S;
    \item binding with 50S. Here we assume that the 50S binds only if the 30S is already binded to form a 70S;
    \item elongation: binding of tRNA (charged or not), binding of ATP and step forward (one step). This step is repeated until the end;
    \item stop (not described here in term of state because on't know how it works).
  \end{enumerate}
\end{enumerate}
Follows an example. Assume that there is already a 70S at the beginning of the aggregation 2.
\begin{center}
  \tiny
  \begin{tabular}{|c|c|c|c|c|c|c|c|c|c|c|c|}
    \hline
    % after \\: \hline or \cline{col1-col2} \cline{col3-col4} ...
    Aggregation & 30S & 50S & non charged tRNA & tRNA 1 & tRNA 2 & tRNA 3 & tRNA 4 & AA 1 & AA 2 & AA 3 & AA 4 \\
    \hline
         2      &  0 & 1 & 0 & 0 & 0 & 0 & 0 & 0 & 0 & 0 & 0 \\
    \hline
  \end{tabular}
\end{center}
Now assume that a tRNA charged with an AA 1 is binded.
\begin{center}
  \tiny
  \begin{tabular}{|c|c|c|c|c|c|c|c|c|c|c|c|}
    \hline
    % after \\: \hline or \cline{col1-col2} \cline{col3-col4} ...
    Aggregation & 30S & 50S & non charged tRNA & tRNA 1 & tRNA 2 & tRNA 3 & tRNA 4 & AA 1 & AA 2 & AA 3 & AA 4 \\
    \hline
         2      &  0 & 0 & 0 & 1 & 0 & 0 & 0 & 0 & 0 & 0 & 0 \\
    \hline
  \end{tabular}
\end{center}
Then elongation is performed. Of course an ATP would be consumed but it is not here to be modeled.
\begin{center}
  \tiny
  \begin{tabular}{|c|c|c|c|c|c|c|c|c|c|c|c|}
    \hline
    % after \\: \hline or \cline{col1-col2} \cline{col3-col4} ...
    Aggregation & 30S & 50S & non charged tRNA & tRNA 1 & tRNA 2 & tRNA 3 & tRNA 4 & AA 1 & AA 2 & AA 3 & AA 4 \\
    \hline
         2      &  0 & 1 & 0 & 0 & 0 & 0 & 0 & 1 & 0 & 0 & 0 \\
    \hline
  \end{tabular}
\end{center}
Now assume that a 30S binds to the binding site, and then a 50S also binds to form a 70S. The state would evolve like follows.
\begin{center}
  \tiny
  \begin{tabular}{|c|c|c|c|c|c|c|c|c|c|c|c|}
    \hline
    % after \\: \hline or \cline{col1-col2} \cline{col3-col4} ...
    Aggregation & 30S & 50S & non charged tRNA & tRNA 1 & tRNA 2 & tRNA 3 & tRNA 4 & AA 1 & AA 2 & AA 3 & AA 4 \\
    \hline
         2      &  0 & 1 & 0 & 0 & 0 & 0 & 0 & 1 & 0 & 0 & 0 \\
         1      &  1 & 0 & 0 & 0 & 0 & 0 & 0 & 0 & 0 & 0 & 0 \\
    \hline
  \end{tabular}
\end{center}
\begin{center}
  \tiny
  \begin{tabular}{|c|c|c|c|c|c|c|c|c|c|c|c|}
    \hline
    % after \\: \hline or \cline{col1-col2} \cline{col3-col4} ...
    Aggregation & 30S & 50S & non charged tRNA & tRNA 1 & tRNA 2 & tRNA 3 & tRNA 4 & AA 1 & AA 2 & AA 3 & AA 4 \\
    \hline
         2      &  0 & 1 & 0 & 0 & 0 & 0 & 0 & 1 & 0 & 0 & 0 \\
         1      &  0 & 1 & 0 & 0 & 0 & 0 & 0 & 0 & 0 & 0 & 0 \\
    \hline
  \end{tabular}
\end{center}
Now we assume that we allow the 70S in aggregation 1 to go to aggregation 2 because we are very happy with this happening (!).
\begin{center}
  \tiny
  \begin{tabular}{|c|c|c|c|c|c|c|c|c|c|c|c|}
    \hline
    % after \\: \hline or \cline{col1-col2} \cline{col3-col4} ...
    Aggregation & 30S & 50S & non charged tRNA & tRNA 1 & tRNA 2 & tRNA 3 & tRNA 4 & AA 1 & AA 2 & AA 3 & AA 4 \\
    \hline
         2      &  0 & 1 & 0 & 0 & 0 & 0 & 0 & 1 & 0 & 0 & 0 \\
         2      &  0 & 1 & 0 & 0 & 0 & 0 & 0 & 0 & 0 & 0 & 0 \\
    \hline
  \end{tabular}
\end{center}
It is possible to be less detailed about the tracking of each ribosome.





\section{Processes}
The minimum requirement is to be able to simulate a cell cycle during. Thus it should model
\begin{enumerate}
  \item DNA: replication and segregation;
  \item mass production: transcription, translation and metabolism. This includes the stringent response via relA;
  \item cell division: cytokinesis (and FtsZ ring?).
\end{enumerate}



\subsection{Simple reaction}
By simple, it is understood that the reaction is performed in one step, instantaneously. The result of a reaction is to change the pools of the reactants and products whenever the reaction happens. {\em When the reaction happens is dealt with in another process}. To describe a reaction, the needed information is thus the pools of the reactants and the enzyme as well as the stoichiometry of the reaction.

\medskip

Let us consider the reaction written in the conventional direction:
$$
  \reactionRev{\underbrace{\sum_{l\in \mathcal{I}_S} m_{l}^SS_{l}}_{Substracts \text{ } S}}{\underbrace{\sum_{l\in\mathcal{I}_P}m_{l}^PP_l}_{Products \text{ } P}}{E}{}.
$$
Note that if this is an enzymatic reaction, then the reaction should not be possible without the enzyme. So that the process is defined by:
\begin{itemize}
  \item inputs:
  \begin{itemize}
    \item the direction of the reaction;
    \item the pool(s) $S_l^{pool}$ of the substract(s);
    \item the pool(s) $E^{pool}$ of the enzyme(s);
    \item the pool(s) $X_l^{pool}$ of the product(s);
  \end{itemize}
  \item parameters: the stoichiometry $m_{l}^S$ and $m_{l}^P$;
  \item processes: if $E^{pool}>0$ (otherwise nothing changes) then change
  \begin{itemize}
    \item if the reaction is in the forward direction: $S_l^{pool} = S_l^{pool} - m_{l}^S$ and $P_l^{pool} = P_l^{pool} + m_{l}^P$;
    \item if the reaction is in the backward direction: $P_l^{pool} = P_l^{pool} - m_{l}^P$ and $S_l^{pool} = S_l^{pool} + m_{l}^S$.
  \end{itemize}
\end{itemize}

\medskip

The above mechanism for process modeling works well as long as every possible 'state' of the reactants are states of the simulator. For instance, we consider the simple reaction (without enzyme for the sake of simplification)
$$
  \reactionIrr{\reactionRev{P_1+P_2}{P_1P_2}{}{}}{Q_1Q_2}{}{}.
$$
In reality, both $P_1$ and $P_2$ have a property, say phosphorized or not. We then have the possible reactions
$$
  \left\{
    \begin{array}{l}
      \reactionIrr{\reactionRev{P_1+P_2}{P_1P_2}{}{}}{Q_1Q_2}{}{} \\
      \reactionIrr{\reactionRev{P_1^*+P_2}{P_1^*P_2}{}{}}{Q_1^*Q_2}{}{} \\
      \reactionIrr{\reactionRev{P_1+P_2^*}{P_1P_2^*}{}{}}{Q_1Q_2^*}{}{} \\
      \reactionIrr{\reactionRev{P_1^*+P_2^*}{P_1^*P_2^*}{}{}}{Q_1^*Q_2^*}{}{} \\
    \end{array}
  \right.
$$
In addition, we can also have the reactions
$$
  \left\{
    \begin{array}{l}
      \reactionRev{P_1 + X}{P_1^*}{}{} \\
      \reactionRev{P_2 + X}{P_2^*}{}{} \\
      \reactionRev{P_1P_2 + X}{P_1^*P_2}{}{} \\
      \reactionRev{P_1P_2 + X}{P_1P_2^*}{}{} \\
      \reactionRev{P_1^*P_2 + X}{P_1^*P_2^*}{}{} \\
      \reactionRev{P_1P_2^* + X}{P_1^*P_2^*}{}{} \\
      \reactionRev{Q_1Q_2 + X}{Q_1^*Q_2}{}{} \\
      \reactionRev{Q_1Q_2 + X}{Q_1Q_2^*}{}{} \\
      \reactionRev{Q_1^*Q_2 + X}{Q_1^*Q_2^*}{}{} \\
      \reactionRev{Q_1Q_2^* + X}{Q_1^*Q_2^*}{}{} \\
    \end{array}
  \right.
$$
Of course, if one puts all the states as described above, the process described above is fine. In this case, the simulator state would be the pools of
$$
  X,\,P_1,\,P_1^*,\,P_2,\,P_2^*,\,P_1P_2,\,P_1^*P_2,\,P_1P_2^*,\,P_1^*P_2^*,\,Q_1Q_2,\,Q_1^*Q_2,\,Q_1Q_2^*,\,Q_1^*Q_2^*
$$
that is 13 states. We want to decrease the number of state. The idea is that the intermediate state are can be computed from the information on $P_1$, $P_1^*$, $P_2$ and $P_2^*$ if we also know the number that are used for these intermediate reactants. Assume, we need the pools of
$$
  X,\,P_1,\,P_1^*,\,P_2,\,P_2^*,\,Q_1Q_2,\,Q_1^*Q_2,\,Q_1Q_2^*,\,Q_1^*Q_2^*.
$$
One would then split in two pools (free and used for $P_1$ and $P_2$ type reactant) the ones of $P_1$, $P_1^*$, $P_2$ and $P_2^*$ leading to also 13 states. Moreover, there is a need for computation to recover the pools of $P_1P_2$, $P_1^*P_2$ , $P_1P_2^*$, and $P_1^*P_2^*$, which is really bad. The relations would have been
$$
  P_{1used}^{pool}+P_{1used}^{*pool} = P_{2used}^{pool}+P_{2used}^{*pool} = (P_1P_2)^{pool}+(P_1^*P_2)^{pool}+(P_1P_2^*)^{pool}+(P_1^*P_2^*)^{pool}
$$
along with
$$
  \left\{
    \begin{array}{l}
      (P_1P_2)^{pool} = \min(P_{1used}^{pool},P_{2used}^{pool}) \\
      (P_1^*P_2^*)^{pool} = \min(P_{1used}^{*pool},P_{2used}^{*pool}) \\
      (P_1^*P_2)^{pool} = P_{2used}^{pool} - (P_1P_2)^{pool} = P_{1used}^{*pool} - (P_1^*P_2^*)^{pool} \\
      (P_1P_2^*)^{pool} = P_{1used}^{pool} - (P_1P_2)^{pool} = P_{2used}^{*pool} - (P_1^*P_2^*)^{pool} \\
    \end{array}
  \right.
$$
Now assume that we only need for the reactant of type $Q_1Q_2$ only the pool of $Q_1Q_2$ separately and that the other pools are not necessary:
$$
  X,\,P_{1free},\,P_{1used},\,P_{1free}^*,\,P_{1used}^*,\,P_{2free},\,P_{2used},\,P_{2free}^*,\,P_{2used}^*,\,Q_1Q_2
$$
We only have 10 states but there is a loss of information as it is now impossible to distinguish between $P_1P_2$ and $Q_1Q_2$ type of reactants. The relation are now
$$
  \left\{
    \begin{array}{l}
      (P_1P_2)^{pool} = \min(P_{1used}^{pool},P_{2used}^{pool}) \\
      (P_1^*P_2^*)^{pool} + (Q_1^*Q_2^*)^{pool} = \min(P_{1used}^{*pool},P_{2used}^{*pool}) \\
      (P_1^*P_2)^{pool} + (Q_1^*Q_2)^{pool} = P_{2used}^{pool} - (P_1P_2)^{pool} = P_{1used}^{*pool} - \min(P_{1used}^{*pool},P_{2used}^{*pool}) \\
      (P_1P_2^*)^{pool} + (Q_1Q_2^*)^{pool} = P_{1used}^{pool} - (P_1P_2)^{pool} = P_{2used}^{*pool} - \min(P_{1used}^{*pool},P_{2used}^{*pool}) \\
    \end{array}
  \right.
$$
Not worth the bother (imagine for 2 properties on $P_1$ and $P_2$ instead of only 1!!!). Use the full information case or limit the number of properties or use the following intermediate solution with the state
$$
  X,\,P_1,\,P_1^*,\,P_2,\,P_2^*,\,Q_1Q_2,\,Q_1^*Q_2,\,Q_1Q_2^*,\,Q_1^*Q_2^*
$$
and with
$$
  (P_1P_2)_{all},\, X_{used}
$$
$(P_1P_2)_{all}$ standing for the whole $P_1P_2,$ $P_1^*P_2$, $P_1P_2^*$ and $P_1^*P_2^*$ and $X_{used}$ standing for $X$ used in $(P_1P_2)_{all}$. A condition for feasibility is then $X_{used}^{pool} \leq 2 (P_1P_2)_{all}^{pool}$. One would have the following possible reactions
$$
  \left\{
    \begin{array}{l}
      \reactionIrr{\reactionRev{P_1+P_2}{(P_1P_2)_{all}}{}{}}{Q_1Q_2}{}{} \\
      \reactionIrr{\reactionRev{P_1^*+P_2}{(P_1P_2)_{all}+X_{used}}{}{}}{Q_1^*Q_2}{}{} \\
      \reactionIrr{\reactionRev{P_1+P_2^*}{(P_1P_2)_{all}+X_{used}}{}{}}{Q_1Q_2^*}{}{} \\
      \reactionIrr{\reactionRev{P_1^*+P_2^*}{(P_1P_2)_{all}+2 X_{used}}{}{}}{Q_1^*Q_2^*}{}{} \\
    \end{array}
  \right.
$$
In addition, we can also have the reactions
$$
  \left\{
    \begin{array}{l}
      \reactionRev{P_1 + X}{P_1^*}{}{} \\
      \reactionRev{P_2 + X}{P_2^*}{}{} \\
      \reactionRev{(P_1P_2)_{all} + X}{(P_1P_2)_{all} + X_{used}}{}{} \\
      \reactionRev{Q_1Q_2 + X}{Q_1^*Q_2}{}{} \\
      \reactionRev{Q_1Q_2 + X}{Q_1Q_2^*}{}{} \\
      \reactionRev{Q_1^*Q_2 + X}{Q_1^*Q_2^*}{}{} \\
      \reactionRev{Q_1Q_2^* + X}{Q_1^*Q_2^*}{}{} \\
    \end{array}
  \right.
$$
The difficulty would be to decide when these reactions can occur. An even 'more' intermediate solution would be to use the states (assuming that $Q_1^*Q_2$ is the reactant of interest since otherwise it is too simple!)
$$
  X,\,P_1,\,P_1^*,\,P_2,\,P_2^*,\,Q_1^*Q_2
$$
and with
$$
  (P_1P_2)_{all},\, X_{Pused},\,(Q_1Q_2)_{all},\, X_{Qused}
$$
with the same constraint as above. The set of reactions are then
$$
  \left\{
    \begin{array}{l}
      \reactionIrr{\reactionRev{P_1+P_2}{(P_1P_2)_{all}}{}{}}{(Q_1Q_2)_{all}}{}{} \\
      \reactionIrr{\reactionRev{P_1^*+P_2}{(P_1P_2)_{all}+X_{Pused}}{}{}}{Q_1^*Q_2}{}{} \\
      \reactionIrr{\reactionRev{P_1+P_2^*}{(P_1P_2)_{all}+X_{Pused}}{}{}}{(Q_1Q_2)_{all}+X_{Qused}}{}{} \\
      \reactionIrr{\reactionRev{P_1^*+P_2^*}{(P_1P_2)_{all}+2 X_{Pused}}{}{}}{(Q_1Q_2)_{all}+X_{Qused}}{}{} \\
    \end{array}
  \right.
$$
In addition, we can also have the reactions
$$
  \left\{
    \begin{array}{l}
      \reactionRev{P_1 + X}{P_1^*}{}{} \\
      \reactionRev{P_2 + X}{P_2^*}{}{} \\
      \reactionRev{(P_1P_2)_{all} + X}{(P_1P_2)_{all} + X_{used}}{}{} \\
      \reactionRev{(Q_1Q_2)_{all} + X}{Q_1^*Q_2}{}{} \\
      \reactionRev{(Q_1Q_2)_{all} + X_{Qused}}{Q_1^*Q_2 + X}{}{} \\
      \reactionRev{(Q_1Q_2)_{all} + X}{(Q_1Q_2)_{all} + X_{Qused}}{}{} \\
    \end{array}
  \right.
$$
Again, the difficulty would be to decide when these reactions can occur.

Finally, it could also be possible to loose even more information by deciding that there only an active form and some non-active forms (one non-active form per type: phosphorized or non-folded -- hum mauvais exemple, pas grave -- for instance). Only the active form can go forward in the pathway; the non-active form must go back into its previous state. Going back to our example with the pathway being:
$$
  \reactionIrr{\reactionRev{P_1+P_2}{P_1P_2}{}{}}{Q_1Q_2}{}{} .
$$
One would consider with this pathway the non-active forms phosphorized and non-folded:
$$
  \left\{
    \begin{array}{l}
      \reactionRev{P_1^{nonFold}}{P_1}{C_1}{} \\
      \reactionRev{P_2^{nonFold}}{P_2}{C_2}{} \\
      \reactionRev{P_1 + X}{P_1^{phosphorized}}{}{} \\
      \reactionRev{P_2 + X}{P_2^{phosphorized}}{}{} \\
      \reactionRev{(P_1P_2)^{nonFold}}{P_1P_2}{C}{} \\
      \reactionRev{P_1P_2 + X}{(P_1P_2)^{phosphorized}}{}{} \\
    \end{array}
  \right.
$$



\medskip

\textcolor[rgb]{1.00,0.00,0.00}{Would be able to model assembly of 30S, 50S and tRNA charging (hum...) if each pool is correctly encoded somewhere as a state.
} 

\subsection{DNA processes}
\subsubsection{DNA replication}

During exponential growth, the cell needs to duplicate its DNA content in order to proceed to division. Replication of DNA needs to be well coordinated with cell growth and division to ensure viability of the daughter cells. There are three important phases: replication initiation, elongation and termination. These phases are controlled by several (partly unknown) mechanisms to adapt to external conditions and adjust growth rate. For example, several bacteria, including \textit{E. coli} and \textit{B. subtilis}, are able to initiate replication several times during one cell cycle, so that the cell cycle can actually be shorter than the time needed to replicate the full chromosome \citep{reyes-lamothe_chromosome_2012}.

\paragraph{Pre-RC assembly} Before initiation can start, a pre-Replication Complex has to be assembled in order to load the replication fork.

\paragraph{Replication initiation} Replication initiation of the chromosome is an event that appears to be precisely controlled. Replication should only be initiated if growth conditions are favorable. What is more, replication should not be triggered several times during cell cycle (except in excellent growth conditions), implying the existence of mechanisms that inhibit replication initiation once replication has already started.

Initiation is mainly controlled by DnaA, a protein that can bind DNA in its activated form, DnaA-ATP. There are numerous DnaA binding boxes along the chromosomes, but only a few of them are essential for replication initiation. The latter are located next to the \textit{oriC} locus (Figure \ref{fig:dnaA}, top), where the replisomes are loaded and replication actually starts. Interestingly, \textit{oriC} and the \textit{dnaA} gene are colocated in numerous bacteria \citep{briggs_chromosomal_2012}, so that the binding of DnaA inhibits its own expression, autoregulating the levels of DnaA available.

When it is activated, DnaA polymerizes along the DnaA binding boxes, unwinding the DNA around \textit{oriC}. It is probable that this unwinding is not necessarily performed by DnaA itself. For example, in \textit{B. subtilis}, DnaA binds with DnaD, which is mainly responsible for untwisting \citep{briggs_chromosomal_2012}. Once the DNA is sufficiently untwisted, a neighboring AT-rich region, termed DNA-Unwinding Element (DUE), opens slightly, enabling loading of the first elements of the replisomes, the helicase and helicase loader (DnaC-DnaI for \textit{B. subtilis}, DnaB-DnaC for \textit{E. coli}). The position of DnaA binding boxes is not exactly conserved in different bacterial species, the loading mechanism is poorly understood. Because a loop is observed during replication initiation of \textit{B. subtilis}, \citet{briggs_chromosomal_2012} propose that DnaD is used for loop-forming and that the loop enables synchronous loading of the two replisomes (one for each direction) through DnaB (Figure \ref{fig:dnaA}, bottom).

\begin{figure}[!ht]
	\centering
	\includegraphics[width=0.8\linewidth]{figure/dnaABindingBoxes}
	\includegraphics[width=0.5\linewidth]{figure/dnaAPolymerizationModel}
	\caption{Control of DNA replication through DnaA proteins and other helper proteins in \textit{B. subtilis}. Around \textit{oriC}, several DnaA binding sites (blue and green triangles) allow for the polymerization of DnaA and, eventually, opening of the DNA at the AT-rich DNA-Unwinding Element (DUE) (top). Model for replisome loading by helper proteins (bottom). DnaD accelerates DnaA binding and its unwinding and polymerization activities form a loop. Finally DnaB is recruited on each end on the loop, each DnaB loading a helicase DnaC on one of the strands with the help of helicase-loader DnaI. Figures from \citet{briggs_chromosomal 2012}.}
	\label{fig:dnaA}
\end{figure}
How initiation is controlled is largely unknown. For example, \textit{E. coli} does not have homologs for \textit{B. subtilis} DnaB and DnaD proteins (\textit{E. coli} DnaB and DnaC are the homologs of \textit{B. subtilis} DnaC and DnaI, respectively). As a result, the initiation is strongly dependent on DnaA-ATP levels in \textit{E. coli} but less in \textit{B. subtilis}, where DnaB and DnaD seem more critical \citep{briggs_chromosomal_2012}. In any case, decreasing DnaA-ATP levels by hydrolysis seems to be an efficient way to prevent initiation. This seems to be one of the strategies to avoid immediate reinitiation. The replisome contains processivity $\beta$-clamps (see below), that bind to proteins that hydrolize DnaA-ATP. What is more, once replication has started, the number of DnaA binding sites rapidly increases, diluting remaining DnaA-ATP. The concentration of DnaA-ATP may increase again when DnaA is newly synthesized. Another possible mechanism for initiation control is the ParAB-\textit{parS} system \citep{reyes-lamothe_chromosome_2012}. This system, probably responsible for chromsome segregation (see below), binds to the \textit{parS} locus through parB. It seems that ParA binds DnaA and influences replication initiation, maybe because ParB hydrolyzes ParA, complexing ParB-\textit{parS} to the \textit{oriC}, this association being stabilized by SMC proteins. The complex can then migrate towards one of the cell poles and it is possible that it may be unavailable for reinitiation during that time.

Even though our understanding of the initiation of replication on the chromosome is limited, experiments show that it is strictly controlled by the cell. On the other hand, the replication of other DNA elements such as plasmids is not as severely controlled, as their exact number within the cell can vary. Initiation is not triggered by DnaA but by other proteins or RNAs that are plasmid-specific. Shortly, there are two main types of plasmids in a bacterial cell: large plasmids present in low copy numbers and small plasmids present in high copy numbers (Figure \ref{fig:plasmidInitiation}). The large plasmids have a similar initiation system as the chromosome, triggered by a DNA binding protein coded by the plasmid itself (generically called Rep). Initiation is probably regulated by the copy number of plasmids, either through a RNA coded by the plasmid that blocks the synthesis of Rep, or simply because Rep is able to polymerize at high concentrations, becoming unavailable for initiation. In small plasmids, initiation is probably controlled by a RNA and DNA polymerase I. The RNA (termed RNAII) slightly opens the double helix and binds to DNA, acting as a primer for DNA polymerase I, which further opens DNA. A single replisome can then be loaded on the other strand (in this case, replication is unidirectional). Copy number is probably controlled by another RNA (termed RNAI), which interfers with RNAII at sufficiently high concentrations.

\begin{figure}[!ht]
	\centering
	\includegraphics[width=0.8\linewidth]{figure/plasmidReplicationInitiation}
	\caption{Plasmid replication follows similar principles as chromosome replication (it uses the same replisome). However, initiation is regulated by plasmid specific elements. Small plasmids probably use RNAs to initiate (RNAII) and inhibit (RNAI) replication. Large plasmids use a protein that acts similarly to DnaA and which is regulated by the plasmid copy number. Figure from \citet{reyes-lamothe_chromosome_2012}}.
	\label{fig:plasmidInitiation}
\end{figure}

\paragraph{Replication elongation} Once replication is initiated, helicases are loaded upon the DNA strands, one in each direction (see above). From the helicases, the whole replisome complex can be loaded and start polymerizing DNA. The replisome ensures processivity and rapid replication of the whole chromosome, as well as synchronous replication of the leading strand and the lagging strand. Indeed, DNA polymerases are only able to assemble DNA in the 5' to 3' sense, which corresponds to the direction the leading strand is processed, but antisense to lagging strand processing. Therefore, the leading strand is easily handled while replication of the lagging strand is handled by loop forming that allows for fragment-wise replication of a few kbs of DNA (termed Okazaki fragment).

The composition of the replication reflects these different tasks (Figure \ref{fig:replisome}). The helicase DnaB (DnaC for \textit{B. subtilis}) is formed of a hexamer that separates the DNA, forming the replication fork. Bound to the helicases, three primases DnaG stabilize the helicase structure and cooperate for synthesis of short RNA sequences on the lagging strand called primers used to initiate Okazaki fragments. Finally, a clamp loader bound to the helicase is responsible for recruiting DNA polymerases and $\beta$-clamps. The number of DNA polymerases can vary depending on the number of subunits $\tau$ in the clamp loader \citep{reyes-lamothe_chromosome_2012,stratmann_dna_2014}: replisomes with 2 or 3 associated DNA polymerases have been observed, the latter seemingly more efficient. The nature of DNA polymerases is also unclear: it seems that in \textit{E. coli} Pol III is preferentially recruited because of its high fidelity, while in \textit{B. subtilis} two different polymerases may be used for the leading and the lagging strand \citep{reyes-lamothe_chromosome_2012,stratmann_dna_2014}. The recruitment of $\beta$-clamps is also essential as it binds to DNA polymerases and increases their processivity from a few nucleotides to several tenths of kb \citep{reyes-lamothe_chromosome_2012}. Because these elements are bound to the clamp loader, they are efficiently recycled, allowing for very fast replication.
\begin{figure}[!ht]
	\centering
	\includegraphics[width=0.6\linewidth]{figure/replisome}
	\caption{Standard bacterial replisome stucture (as reconstructed from \textit{E. coli}). Figure from \citet{stratmann_dna_2014}.}
	\label{fig:replisome}
\end{figure}

Coordination of leading and lagging strand replication is not fully understood. As replication occurs continuously on the leading strand, it seems intuitive that replication should occur more rapidly than on the lagging strand or that the loop forming/relase of Okazaki fragments has to be particularly efficient. Two models have been proposed for the loop forming process: the collision process and the signaling process (Figure \ref{fig:replisomeElongation}, left). In both models, DNA polymerase starts from a primer, progressively forming a loop as ssDNA (protected by SSB) and recently polymerized DNA accumulate between the fork and the DNA polymerase. In the collision model, the DNA polymerase proceeds until the next Okazaki fragment, stalling and becoming available for elongation from the next primer. If a primer is ready before the current fragment is finished, the fork could pause, making this process quite inefficient. In the signalling model, the DNA polymerase is reused as soon as a primer is ready, so that a lot of fragments remain incomplete, containing large portions of ssDNA. In fact, the presence of 3 polymerases indicates that the reality might be in-between, explaining how the lagging strand synthesis can be as efficient as the leading strand synthesis (Figure \ref{fig:replisomeElongation}) \citep{stratmann_dna_2014,duderstadt_replication-fork_2014}. With 3 polymerases, 2 polymerases might be synthesizing at the same time, one finishing the previous fragment while the other one is available for recruitment on a new primer. As a matter of fact, it is also highly probable that the replisome is very dynamic, with frequent recruitment and detachment of primases and polymerases. In this way, even a 2 polymerase replisome could operate in a similar manner, by periodically detaching polymerases to finish a fragment and recruiting a new polymerase at the replisome. This remains to investigate but seems pretty likely as numerous polymerases gravitate around the replisome and such exchanges have been observed on the leading strand \citep{stratmann_dna_2014}.
\begin{figure}[!ht]
	\centering
	\includegraphics[width=0.49\linewidth]{figure/replisomeElongation}
	\includegraphics[width=0.49\linewidth]{figure/replisomeThreePolymerases}
	\caption{Elongation models for the lagging strand (left). Likely elongation according to recent experiments that mixes the two previous models (right). Figure from \citet{stratmann_dna_2014}.}
	\label{fig:replisomeElongation}
\end{figure}

Some details of elongation remain unclear. RNA primers have to be removed, resynthesized and ligated by specific polymerases, but the coordination with the replisome has not been investigated (to our knowledge). The cooperation with SMC molecules or obstacle management is known to exist but very little is known. Some elements are presented in the repair section, where stalling of the replisome is handled by RecA and restarting of replication is handled by a specific primosome complex. However, it is unknown if the replisome collapses, helps recruiting repair proteins or lower fidelity polymerases that can bypass some specific DNA damages.

\paragraph{Replication termination} Termination of replication occurs in the \textit{terC} region thanks to Tus proteins. These proteins are able to block a replisome along one direction. Replisomes will thus meet along a small segment delimited by two Tus proteins in opposite directions, probably even at the site of one of these proteins. The length of the two replicons is thus not totally fixed but limited to a certain range.


\paragraph{Computational representation}
See illustration how the cell chromosome is used. Some more information may be needed TBD.
When a chromosome is fully duplicated (when the first column of the cell chromosome only contains -2), the process consists of deleting the current chromosome and creating 2 new ones with the correct initialization. TBD do we clean the chromosomes? Typically, when the chromosome was manipulated.

Needs also other things but from the DNA point of view, there seems to have enough information.

\subsubsection{DNA movement}
\textcolor[rgb]{1.00,0.00,0.00}{What do we do when a gene change of volume due to condensation or segregation for example. Normally, a matter of changing the number of the volume in the cell chromosome and the corresponding volume chromosomes. But assume that anything that is currently binded to the DNA is the property of the DNA and was 'erased' from everywhere else. Also assume that volume chromosome only contains the strictly necessary information about the DNA in the volume and that it is a state with changing size.}

\subsubsection{DNA manipulations}
\paragraph{Codon aggregation damage}
\textcolor[rgb]{1.00,0.00,0.00}{Gap site, Abasic site, Sugar-phosphate, Base, Intrastrand cross link, Strand break, Holliday junction}
DNA is subject to numerous forms of damage that can be either endogenous or exogenous. They can result in chemical modifications of some bases, in single strand breakage (missing of one base on one of the strands), double strand breakage or cross links between DNA strands. Chemical modifications can originate from different type of radiations (UV, x-rays, etc.), drugs or reactants naturally present in the cell leading to alkylations, oxydations, deaminations, etc. Another important source of DNA mismatches is the replication machinery itself which can make use of the wrong dNTP (dUTP for example).

DNA modification is one of the prerequisites to evolution. DNA mutations lead to the development of novel functions, regulatory systems, etc. However, replication fidelity is also essential for selection and conservation of important existent functions. There is a trade-off between these two aspects that is well illustrated in \textit{B. subtilis} by the existence of efficient repair mechanism on the one hand and some DNA polymerases that favor propagation of some types of damage (such as DnaE) on the other hand.



\paragraph{Codon aggregation insertion}
Insertion of one (or several) lines in the DNA states.

\paragraph{Codon aggregation deletion}
Putting 0 in the corresponding places and not deletion of the line(s) because a codon aggregation can be deleted in a fork.

\paragraph{Codon aggregation repair}
There are several pathways employed for DNA repair corresponding to the type of damage undergone.

\subparagraph{Mismatch Repair (MMR)} This pathway is dedicated to reparation of base mismatches. In \textit{E. coli}, repairing is initiated by MutS (Sensor) which detects the mismatch and MutL (Linker) which recruits further proteins. The endonuclease MutH then nicks the DNA next to the mismatch enabling the helicase and exonuclase UvrD to remove bases around the mismatch. The resulting gap is filled in by DNA polymerase III and DNA Ligase. The newly synthesized strand is specifically targeted by these proteins thanks to the methylase Dam used for marking the original strand. In \textit{B. subtilis}, only MutS and MutL seem to be conserved (MutL having an additional endonuclease activity). Recognition of the newly synthesized strand could be linked to a strong coupling of MMR with replication and colocalization with the replisome.

\subparagraph{Base excision repair (BER)} The BER pathway repairs most non-bulky base modifications such as oxydations, deaminations, UTP incorporation, etc. Schematically, glycosylases detect the lesion, remove the damaged base, endonucleases then nick the DNA next to the missing base so that exonucleases remove some bases on the strand around the missing base. The small gap is then closed by a repair DNA polymerase (such as Polymerase I) and ligated by a DNA ligase. For example, in \textit{B. subtilis}, the glycosylases MutM and MutY (part of the GO system) detect oxidized Guanine to avoid its pairing up with dATP. Another example is Uracil DNA-glycosylase, which removes dUMP from DNA. 

\subparagraph{Nucleotide excision repair (NER)} The NER pathway is very similar to the BER pathway, except that it repairs bulky lesions caused by UV radiations or drugs. This pathway is highly conserved and partly regulated by the SOS response (mediated by RecA). The UvrABC complex is responsible for detecting the damaged base and nicking the DNA at surrounding bases. Helicase UvrD removes the nicked segment. Finally, DNA Pol. I and DNA ligase restore the missing segment.

\subparagraph{Alkylation damage} There are specific pathways that address alkylation (such as methylations). \textit{B. subtilis}, as a soil-living bacterium, is particularly exposed to alkylating agents. There are at least three pathways responsible for repairing alkylated bases: two pathways based on glycosylases (one being constitutive, the other inducible) and one pathway based on alkyltransferases (enzymes that suicide by transferring the alkyl group onto themselves).

\subparagraph{Homologous recombination (HR)} During replication, double strand breaks (DSB) can be repaired by using the other copy of the chromosome (Figure \ref{fig:HR}). First, the DSB is stabilized by RecN and digested by protein complexes (AddAB and RecQSJ in \textit{B. subtilis}) that create hangovers of single stranded DNA (ssDNA) on the 3' strands. RecA binds to the ssDNA (probably with the help of the RecFOR complex that prevents SSB from binding). RecA, activated by ATP or dATP, enables invasion of the sister chromosome at a homologous sequence, creating a D-loop where one of the broken 3' strands is inserted and forming Holliday Junctions between the two chromosomes. DNA elongation occurs from the 3' strand and the Holliday Junctions are cleaved by RecU (or a homologous protein).

There is a variation if the DSB caused the replication to stall (Figure \ref{fig:HR}, right). This situation may occur if a single-strand break is present on the original chromosome which becomes a DSB after passage of the replication fork. In this case, there is only one DNA end to digest and the invasion leads to only one Holliday Junction. The primosome complex, composed of Pri and Dna proteins, detects the D-loop and helps loading the replisome to resume replication. It seems that several Structural Maintenance of Chromosome (SMC) proteins are involved throughout the process (such as RecN), but their role is not clearly elucidated.

\begin{figure}[!ht]
\centering
\includegraphics[width=0.54\linewidth]{figure/HR1}
\includegraphics[width=0.44\linewidth]{figure/HR2}
\caption{Homologous recombination and repair of DSB in \textit{B. subtilis} in the general case (left) and when the DSB appears in the replication fork (right). From Lenhart \textit{et al.}.}
\label{fig:HR}
\end{figure}

\subparagraph{Nonhomologous end joining (NHEJ)} This pathway is also responsible for DSB repairing but it is less efficient than homologous recombination. It is used when there is no other copy of the chromosome present in the cell. As in HR, a protein (probably YkoV for \textit{B. subtilis}) binds the DSB and favors recruitment of another protein (LigD like, probably YkoU for \textit{B. subtilis}) that is able to perform exonucleation, polymerisation and ligation. The mechanisms are not totally clear but it seems that because LigD is able to perform these 3 functions, no other protein is needed. However, some bases need to be deleted and repolymerized during the reparation, possibly leading to DNA losses or insertions, making NHEJ a low-fidelity repair mechanism.


\subsubsection{DNA compaction}
DNA compaction occurs through various proteins that are able to clamp the DNA together, forming high-densitiy bundles, leading to a compact form within the cell called a nucleoid. The compaction is due to supercoiling generated by DNA gyrase and topoisomerases, the action of histone-like proteins (HU, IHF, Fis H-NS) and SMC proteins (SMC-ScpA-ScpB for \textit{B. subtilis}, MukBEF for \textit{E. coli})(Fig. \ref{fig:compaction}). According to several studies, the nucleoid adopts a large helical structure composed of two intertwined branches, at least during G1 phase (Fig. \ref{fig:compaction}) \citep{berlatzky_spatial_2008, ptacin_chromosome_2013, fisher_four-dimensional_2013}. This structure is dynamical, with specific loci moving of about 10\% of cell length within a few seconds \citep{wiggins_strong_2010,fisher_four-dimensional_2013} and seems to be ATP-dependent \citep{fisher_four-dimensonal_2013,weber_nonthermal_2012}. \citet{fisher_four-dimensional_2013} identify these variations as waves that could be used to unbind some of the DNA binding proteins, avoiding overcondensation, unwanted linkage between loci and facilitating segregation. In general, it seems that the chromosome is linearly organized around the \textit{oriC}, meaning that the bundling occurs progressively, so that the position within the cell reflects the position along the chromosome \citep{wiggins_strong_2010}.
\textcolor[rgb]{1.00,0.00,0.00}{Condensation, ”clamping” of the DNA by structural maintenance of chromosome (SMC) proteins, supercoiling, macromolecular crowding, charge neuralization?}
If spatialization is used, condensation and segragation might be modelled directly. Supercoiling needs another state.
\textcolor[rgb]{1.00,0.00,0.00}{Compactation should also impact the accessibility of the chromosome.}

\begin{figure}[!ht]
	\centering
	\includegraphics[width=0.4\linewidth]{figure/SMC.png}
	\includegraphics[width=0.4\linewidth]{figure/compaction.png}
	\caption{Proteins responsible for chromosome compaction (left) and model for nucleoid organization in \textit{C. crescentus} (right). Figures from \citet{wang_organization_2013}.}
	\label{fig:compaction}
\end{figure}

\subsubsection{DNA segregation}
As replication proceeds, sister chromosomes/plasmids have to be separated in order to allow proper cell division. This process is also largely unknown, but several models based on experiments have been proposed, it seems that there is no universal solution valid for all bacteria. There are two main challenges: unlinking the chromosomes and driving them to opposite poles of the cell.

Unlinking is generally done at the replisome level. Supercoiling accumulates at the front of the helicase due to its unwinding activity. This supercoiling can be physically propagated to the back of the replisome, creating an entanglement between sister chromosomes. The DNA gyrase limits this propagation by diminishing supercoiling at the front of the replisome, while Topoisomerase IV disentangles the chromosome copies \citep{reyes-lamothe_chromosome_2012}.

Segregation can be done according to several mechanisms, particularly for plasmids. For high copy number plasmids, it is possible that segregation occurs purely through diffusion. For other plasmids, elements of the cytoskeleton can be used to separate the copies (Figure \ref{fig:dnaMigration}ab). ParR proteins may bind to \textit{parC} loci on the plasmid and serve as a basis for actin-like ParM that polymerizes between the copies, progressively seperating them. Similarly, TubR might bind to \textit{tubC} and migrate along filaments of tubulin-like TubZ. The last mechanism may target plasmids as well as the chromosomes (Figure \ref{fig:dnaMigration}c). It is also composed of a DNA binding protein ParB, binding to \textit{parS} (close to \textit{oriC}), and a motor protein ParA. However, ParA attracts ParB only in its activated and DNA-binding form ParA-ATP, probably located along the nucleoid. ParB hydrolises ParA-ATP, releasing it in the cytosol until it gets reactivated and rebinds DNA away from ParB. In this way the two ParB-\textit{parS}-\textit{oriC} complexes migrate in opposite directions until steady-state is reached, with equivalent ParA-ATP pools located on each side of each complex (approximately at the quarter of each pole). It seems that this system works cooperatively with SMC proteins, but the details are yet unknown \citep{reyes-lamothe_chromosome_2012}. Another possibility is radial stress \citep{fisher_four-dimensional_2013}. Sister chromatides are bundled separately but hold together by some tether. They accumulate at the center of the cell, along with the mother DNA, which is bundled separately. The mother DNA pushes the two growing chromatides aside with a force that gets stronger for sterical reasons until the tethers break and the chromatides migrate towards the poles. According to \citet{fisher_four-dimensional_2013}, this phenomenon happens up to four times during segregation. The main idea here is that segregation results from efficient bundling of neosynthesized DNA, so that mother DNA and sister chromatides form 3 very distinct structures that repel each other but are linked by tethers.
\begin{figure}[!ht]
	\centering
	\includegraphics[width=0.8\linewidth]{figure/DNAmigration}
	\caption{Migration of plasmids and chromosome can be mediated by different systems. It can be based on elements of the cytoskeleton: separation through polymerization of actin-like proteins (a), migration along tubulin-like proteins (b). Alternatively, migration may based on an oscillatory system, where activated proteins (ParA-ATP) attracts another protein (ParB) linked to the plasmid or chromosome (\textit{parS} loci) that hydrolises it (c). DNA migration is then controlled by the location of pools of activated proteins. Figure from \citet{reyes-lamothe_chromosome_2012}.}
	\label{fig:dnaMigration}
\end{figure}

Similar to initial segregation, final segregation includes decatanation of two chromosomes and migration of the \textit{ter} region to the two poles, but it is assisted by a new protein and associated with the formation of the FtsZ ring at the septum. The FtsZ ring cannot polymerize as long as the nucleoid is located at the center of the cell. Cytokinesis thus begins when the initial migration of DNA copies is already advanced. Once FtsZ ring forms, the DNA translocase FtsK (SftA in \textit{B. subtilis}) is recruited by the divisome to the membrane next to the FtsZ ring. FtsK seems to coordinate several actions in the final segregation. FtsK binds to \textit{dif} loci next to the \textit{ter} regions, aligning the \textit{ter} region with the septum. FtsK can also translocate remaining DNA at the final stages of septum closing. FtsK also cooperates with Xer proteins to separate chromosomes copies that might have merged due to recombination by creating a Holliday junction.
 

\subsubsection{DNA transformation}

DNA transformation is the incorporation of extracellular DNA into the cell. The role of transformation is not totally understood. Most probably, it allows for generation of bacterial sex and generation of an enhanced genetic diversity in changing environments or stress situation. The master regulator of competence ComK induces more than 150 genes, 30 of which only are known to be directly involved in DNA transformation.

Transformation uses the recombination machinery to integrate the DNA into its own genetic material. Numerous proteins of the SOS response are therefore involved but it seems that the induction of these proteins is mediated by ComK, the master regulator of competence state, and is LexA-independent~\citep{kidane_cell_2012}. DNA transformation is only activated in specific physiological conditions by quorum sensing. ComX and the Competence and Sporulation Factor are secreted, activating ComP-ComA transduction. More than 150 genes are then directly or indirectly induced, including the DNA translocation machinery and the recombination machinery~\citep{kidane_cell_2012}.

The DNA translocation machinery assembles towards the cell pole and is composed of three groups \citep{kidane_cell_2012}:
\begin{itemize}
	\item The first group is responsible for binding dsDNA and transforming it into ssDNA. Known proteins are from the ComG family (ComGA, ComGC, ComGD, ComGE, ComGG) that form a pseudopilus able to drag DNA toward the membrane and NucA, which fragments dsDNA, allowing it to be digested to ssDNA by an unknown enzyme.
	\item The second group translocates ssDNA into the cell. It is composed of proteins from the ComE and ComF families (ComFA, ComEA, ComEC, maybe ComFB, ComFC, ComEB). The junction with the first group is done by ComFA and ComGA.
	\item The last group binds DNA inside the cell and performs its integration. It is composed of the induced proteins DprA, SsbA, SsbB, CoiA, RecA and the constiutive proteins RecU and RecX. Its association with the rest of the translocation machinery is highly dynamic.
\end{itemize}
Once the ssDNA is translocated within the cell, the third group protects it from degradation and recruits protein that are also involved in classical homologous recombination:
\begin{itemize}
	\item SsbA and SsbB bind and protect ssDNA from degradation.
	\item RecN and SsbE stabilize the ends of the DNA.
	\item DprA and RecO destabilize the SSB proteins and allow RecA loading.
	\item RecA loading is regulated by RecF, RecX and RecU.
\end{itemize}
Once RecA is loaded along the ssDNA, it forms a compact filament that is driven towards the center of the cell. It may be integrated within the genome if it is homologous to an endogeneous sequence (with a probability of up to 40\%~\citep{kidane_cell_2012}) or may be transformed into a plasmid if it has a primosome assembly site (\textit{pas})~\citep{kidane_cell_2012}. If a homologous sequence is found, an invasion similar to Homologous Recombination occurs (see DNA repair process). However, branch migration is not fully understood and HJ resolution may not be mediated by RecU, contrary to classical homologous recombination.

Formation of new plasmid is a little more complex as it may occur according to different mechanisms that are species specific. \citet{kidane_cell_2012} list three possibilites that could contribute to plasmid creation for ssDNA containing a \textit{pas} sequence:
\begin{itemize}
	\item Plasmid facilitation: the ssDNA transiently binds to the chromosome, forming a loop that is replicated using the free ends of the ssDNA as primers, closing the loop by adding chromosomal DNA to the ssDNA.
	\item Plasmid transformation (necessitates internal homologies): in the absence of homology, RecA unbinds and the ssDNA is converted to dsDNA thanks to his \textit{pas}. It is then circularized based on its internal homologies.
	\item Monomeric activation: poorly understood process that could work even without internal homologies. Does not seem to exist in \textit{B. subtilis}.
\end{itemize}


\subsection{Transcription}
%The transcription for prokaryote is composed of 4 steps:
%\begin{enumerate}
%  \item promoter search;
%  \item initiation;
%  \item elongation;
%  \item termination.
%\end{enumerate}




\subsubsection{Promoter search} Before binding to DNA, the RNA polymerase has to find it: it is called promoter search. It was generally admitted that a 3D-diffusion search was not enough as some search rate measure were above the 3D-diffusion possible rate. Some other mechanisms were proposed: 1D sliding along the DNA, 1D hoping and DNA intersegment jump. In \citep{Hal:09}, it is suggested however that the promoter search rate is consistent with a 3D diffusion rate.



\subsubsection{Initiation}
%\paragraph{Biological process}
The purpose of initiation is to locate precisely the Pribnow's box (TATAAT for $\sigma^{70}$ sigma factor) so that the RNA polymerase can bind to it. This binding is favoured by a sigma factor. A non-bounded RNA polymerase is composed of 4 sub-units ($2\alpha$, $\beta$, $\beta'$ and $\omega$): it is a core-enzyme. The initiation ends with the promoter clearance and eventually the release of the sigma factor. It follows the steps \citep{SdH:11}:
\begin{enumerate}
  \item a sigma factor binds to the core-enzyme becoming a holo-enzyme:
    $$
      \reactionRev{pRNA + \sigma^i}{pRNA\Sigma^i}{}{}
    $$
    There are several types of sigma factor and each of them increases the affinity of the holo-enzyme RNA polymerase to the specific promoter.
  \item The holo-enzyme then binds to a DNA promoter and forms a closed complex. This triggers a series of conformational changes collectively called 'izomerization':
      \begin{itemize}
        \item opens $~13$ bp from the -10 elements beyond the transcription start site for the sigma factor and -35 for the $\alpha$ sub-unit while the complex protects a 'footprint' of around 30 bp from nuclease digestion \citep{vHi:98};
        \item creates the initiation 'bubble' and a stable open complex after an unstable one.
      \end{itemize}
      It is summarized with the reactions:
    $$
      \left\{
        \begin{array}{l}
          \reactionIrr{pRNA\sigma^i + DNA}{cDpRNA\sigma^i}{}{} \\
          \reactionIrr{cDpRNA\sigma^i}{oDpRNA\sigma^i}{}{} \\
        \end{array}
      \right.
    $$
  \item The open complex then synthesizes nascent RNA and tries to leave the promoter site (promoter clearance). However, around 10 abortive initiations happens \citep{GoN:09} in mean before promoter clearance is really performed:
      $$
        \left\{
          \begin{array}{l}
            \reactionIrr{oDpRNA\sigma^i + \sum m_irNTP^i}{oDpRNA\sigma^i + nRNA}{}{} \\
            \reactionIrr{oDpRNA\sigma^i + \sum m_irNTP^i}{DpRNA\sigma}{}{} \\
          \end{array}
        \right.
        .
      $$
      The first reaction models the abortive initiation as the synthesis of the nascent RNA being apart from the DNA - RNA polymerase complex. Physicality, the nascent RNA is still inside the complex and goes away from the complex when the sigma factor is released. The second reaction models the promoter clearance with the nascent RNA still attached to the complex.
  \item The sigma factor is released. Physically, the sigma factor can be either released during the promoter clearance or during the beginning of elongation.
      $$
        \reactionIrr{DpRNA\sigma}{DpRNA}{}{} \\
      $$
      The trigger is not clear however it is released typically when the nascent RNA reaches a length of 12-15 nt.
\end{enumerate}

%\paragraph{Biological error} The authors are not aware of any biological error that can happen during initiation of transcription.







\subsubsection{Elongation}
%\paragraph{Biological process}
The elongation consists in binding the ribose NTP to each other and step forward:
\begin{enumerate}
  \item recruitment of the ribose NTP corresponding:
    $$
      \reactionRev{DpRNA + rNTP^i}{DpRNA^i}{}{} . \\
    $$
    A 'wrong' ribose NTP could be recruited at this step, or even a deoxyribose NTP [reference needed];
  \item binding of the recruited ribose NTP to the former one, catalyzed by a pair of \ce{Mg2+}:
    $$
      \reactionIrr{DpRNA^i + \ce{H2O}}{DpRNA^+ + PP_i}{\ce{Mg^2+}}{} ; \\
    $$
  \item translocation where the RNA polymerase step one base forward:
    $$
      \reactionIrr{DpRNA^+}{DpRNA}{}{} . \\
    $$
\end{enumerate}
The movement of the polymerase forms a Brownian ratchet motion. The elongation rate is around 6.2-20 bp/s. In competition to the step forward motion, there are:
\begin{itemize}
  \item pausing consists in the RNA polymerase to pause during some time but the mechanics are unclear. Promoter proximal pause also happens. \citep{LaW+:14} proposes a consensus sequence of 11 nt length for pausing detection. An early release of the nascent RNA transcript may happen during pausing;
  \item backtracking consists in the cleavage of 2 or more nucleotides. It seems to be a kind of proofreading. It is not clear when and how it happens;
  \item stalling consists in the RNA polymerase to wait, especially for rare ribose NTP to be recruited. Stalling is thought to help the folding of the nascent polypeptide chain [reference needed].
\end{itemize}


%\paragraph{Biological error} An early release of the nascent RNA transcript may happen, especially during pausing. Not the good rNTP



\subsubsection{Termination}
%\paragraph{Biological process}
For prokaryotes, there are two types of termination:
\begin{enumerate}
  \item Rho-independent or independent termination \citep{GuN:99,WaG:10}: nascent RNA forms a rich G-C hairpin followed by 7-9 U bases. This weakens the DNA - RNA polymerase complex. The force due to the hairpin is not enough though and some other mechanisms, which is performed by the binding of NusA, is needed \citep{HeB:08};
  \item Rho-dependent termination: a rho co-factor helps the termination. Two major models exist: $(i)$ the rho factor binds to the rut (rho utilization site of about 70 - 100 nucleotides) or rho binding site and then moves forward towards transcription stop point (tsp) region creating an hairpin; $(ii)$ the rho factor binds to the polymerase \citep{EDWN:10} and an hairpin is created while the polymerase elongates. In the tsp region, there are (potentially) several pause positions for the RNA polymerase in the absence of rho co-factor, the termination occurs at these stop positions.
\end{enumerate}


%\paragraph{Biological error} The authors are not aware of any biological error that can happen during termination of transcription.



\subsubsection{Others}

\paragraph{Transcript slippage} The phenomenon is illustrated in Figure \ref{fig:slippage} in the case where the RNA polymerase idles; forward and backward slippage are also described in \cite{ATM:10}.
\begin{figure}[!ht]
	\centering
	\includegraphics[width=0.8\linewidth]{figure/transcriptSlippage}
	\caption{Illustration of transcript slippage, from \cite{ATM:10}}
	\label{fig:slippage}
\end{figure}
It mainly occurs on homopolymeric tracts. Slippage can occur during initiation, elongation and termination, as well as during replication and translation. Slippage during transcription initiation plays a role in transcription regulation by abortive transcripts \cite{Tur:11}. Bacteria can take advantage of slippage in few ways as described in \cite{ATM:10}:
\begin{itemize}
  \item use a single mRNA to encode two proteins;
  \item restore a reading frame that would be otherwise be interrupted by the addition or deletion of a nucleotide.
\end{itemize}


\subsection{Translation}
See mRNA state for evolution.

\subsection{Stringent response}
\paragraph{Biology} The stringent response is a fundamental regulation of the bacteria as it determines its response to environmental condition such amino acid starvation and stress. It determines the growth rate and acts on transcription and translation.

 \begin{description}
   \item[Transcription] it affects the ability of the polymerase to bind to the DNA for initiation;
   \item[Translation] it inhibs the ribosome  
 \end{description}
 It is based on the the inhibition of the ribosomes by RelA depending on the abundance of amino acids in the cytosol.

\subsection{Metabolism}
\textcolor[rgb]{1.00,0.00,0.00}{\textbf{\emph{\huge Joker!}}}

\subsection{Volume growth}
\input{process/volume.tex}

\subsection{Exchanges}
\subsubsection{Exchange with extracellular conditions}
Not described but a matter of changing the pool (concentrations probably for extracellular conditions) in volume and extracellular conditions. Works in both ways if needed.

\subsubsection{Exchange between volumes}
\textcolor[rgb]{1.00,0.00,0.00}{Not described but a matter of changing the pool in volumes. Works in both ways if needed. Just remind that for 'complex' entities, it is easier if everything is contained in one entity only. I mean, if a mRNA binded with a 70S and an AA 1 charged tRNA moves, then if only one entity contains all these information, it is much easier. Here we would have chosen the mRNA.} 

\subsection{Volume division}
\textcolor[rgb]{1.00,0.00,0.00}{Need to obtain a storage for interface between volume. But otherwise, volume division is a matter of deleting a volume, creating 2 (3? 4?) volumes, adjusting the interfaces and pools. Needs to be a cell process since it can change the chromosome location.} 

\subsection{Cytokinesis, cell fission?}

\begin{figure}[!ht]
	\centering
	\includegraphics[width=0.9\linewidth]{figure/cytokinesis}
	\caption{Location of the FtsZ ring could be mediated by Nucleoid Occlusion (NO) molecules. These proteins bind on the first two thirds of the chromosome (depleted in the \textit{ter} region) and inhibit FtsZ polymerization. As long as segregation has not started, NO molecules inhibit FtsZ polymerization. Once chromosomes have migrated and the center of the cell only contains the unreplicated \textit{ter} region, FtsZ polymerization and septum formation start. Figure from \citet{ptacin_chromosome_2013}}
	\label{fig:cytokinesis}
\end{figure}

\textcolor[rgb]{1.00,0.00,0.00}{Depending on how it is modeled. The part on geometry is 'simple' to model:
\begin{itemize}
  \item volume division;
  \item changing the characteristics of the concerned volumes to adjust the surface of exchange for example.
\end{itemize}
Other part? Needs to be a cell process.}

\subsection{Processes from wholeCell}
Police barr\'{e}e = un process de wholeCell que j'ai remis en haut, sans pour autant l'avoir mod\'{e}liser ou avoir les informations n\'{e}cessaires dans les \'{e}tats.
\begin{itemize}
  \item \sout{Chromosome condensation: DNA clamping by SMC complexes}
  \item \sout{Chromosome segregation}
  \item \sout{Cytokinesis: pinching of the cell membrane}
  \item \sout{DNA damage: Gap site, Abasic site, Sugar-phosphate, Base, Intrastrand cross link, Strand break, Holliday junction}
  \item \sout{DNA repair}
  \item \sout{DNA supercoiling}
  \item FtsZ polymerization
  \item Host interaction: des trucs mais en quoi c'est utile ?
  \item Macromolecular complexation: models the formation of all macromolecular complexes except the 30S and 50S ribosomal particles, the 70S ribosome, the FtsZ ring, and the oriC DnaA complex
  \item \sout{Metabolism: models the import of extracellular nutrients and their conversion into macromolecule building blocks}
  \item Protein activation:  implements a Boolean model of their effects on the functional state – enzymatically competent or incompetent – of mature proteins
  \item Protein decay: models the degradation of protein monomers, macromolecular complexes, cleaved signal sequences, and prematurely aborted polypeptides as well as the misfolding and refolding of protein monomers and complexes
  \item \sout{Protein folding}
  \item \sout{Protein modification: models protein covalent modification including phosphorylation, lipoyl transfer, and α-glutamate ligation}
  \item \sout{Protein processing I: models N-terminal formylmethionine deformylation and N-terminal methionine cleavage, the first steps in post-translational processing. What's that???}
  \item \sout{Protein processing II: models the third step of post-translational processing: lipoprotein diacylglyceryl adduction and lipoprotein and secreted protein signal peptide cleavage. What's that?}
  \item \sout{Protein translocation: models membrane and extracellular protein localization, the second step in post-translational processing}
  \item \sout{Replication}
  \item Replication initialization: determines when during the cell cycle chromosome duplication begins. Uses the protein DnaA (MG469)
  \item \sout{Ribosome assembly:  models the enzyme-catalyzed formation of 30S and 50S ribosomal particles}
  \item \sout{RNA decay: decays all species of RNA, and at all maturation states including aminoacylated states}
  \item \sout{RNA modification: the exact role of rRNA modification is unknown. This process models tRNA and rRNA modification}
  \item \sout{RNA processing:  models operonic RNA cleavage into individual RNA gene products. Something about operons.}
  \item Terminal organelle assembly: models the assembly of the protein content of the terminal organelle
  \item \sout{Transcription: For simplicity, our model doesn’t represent this phenomenon, allowing translation only of completed mRNAs}
  \item Transcription regulation: models the binding of transcriptional regulators to promoters and the fold-change effect of transcriptional regulators on the affinity of RNA polymerase for individual promoters.
  \item \sout{Translation}
  \item \sout{tRNA aminocylation}
\end{itemize}


\end{document}
% ----------------------------------------------------------------
