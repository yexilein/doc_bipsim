A mRNA has the same kind of structure as an operon. \textcolor[rgb]{1.00,0.00,0.00}{Add binding site (and a stop site?) to Chromosome characteristics.}
\begin{center}
  \tiny
  \begin{tabular}{|c|c|c|c|c||c|c|c|c|c|c|c|}
    \hline
    % after \\: \hline or \cline{col1-col2} \cline{col3-col4} ...
    Codon & AA 1 & AA 2 & AA 3 & AA 4 & Aggregation & Binding site & Stop site & AA 1 & AA 2 & AA 3 & AA 4 \\
    \hline
      1   &  0   &  0   &   0  &   0  &       1     &      1       &     0     &   0  &  0   &   0  &   0  \\
    \hline
      2   &  0   &  0   &   1  &   0  &  \multirow{3}{*}{2} & \multirow{3}{*}{0} & \multirow{3}{*}{0} & \multirow{3}{*}{1} & \multirow{3}{*}{0} & \multirow{3}{*}{2} & \multirow{3}{*}{0} \\
      3   &  1   &  0   &   0  &   0  & & & & & & & \\
      4   &  0   &  0   &   1  &   0  & & & & & & & \\
    \hline
      5   &  0   &  0   &   0  &   1  &  \multirow{2}{*}{3} & \multirow{2}{*}{0} & \multirow{2}{*}{0} & \multirow{2}{*}{1} & \multirow{2}{*}{0} & \multirow{2}{*}{0} & \multirow{2}{*}{1} \\
      6   &  1   &  0   &   0  &   0  & & & & & & & \\
    \hline
      7   &  0   &  0   &   0  &   0  &      4      &      0       &     1     &   0  &  0   &   0  &   0  \\
    \hline
  \end{tabular}
\end{center}
So an mRNA is represented by
\begin{enumerate}
  \item what it produces (how this information is stored?);
  \item the aggregated table of information;
  \item per aggregation, the state. Basically, for a brief description, we assume that the translation is performed in the following steps
  \begin{enumerate}
    \item binding with 30S;
    \item binding with 50S. Here we assume that the 50S binds only if the 30S is already binded to form a 70S;
    \item elongation: binding of tRNA (charged or not), binding of ATP and step forward (one step). This step is repeated until the end;
    \item stop (not described here in term of state because on't know how it works).
  \end{enumerate}
\end{enumerate}
Follows an example. Assume that there is already a 70S at the beginning of the aggregation 2.
\begin{center}
  \tiny
  \begin{tabular}{|c|c|c|c|c|c|c|c|c|c|c|c|}
    \hline
    % after \\: \hline or \cline{col1-col2} \cline{col3-col4} ...
    Aggregation & 30S & 50S & non charged tRNA & tRNA 1 & tRNA 2 & tRNA 3 & tRNA 4 & AA 1 & AA 2 & AA 3 & AA 4 \\
    \hline
         2      &  0 & 1 & 0 & 0 & 0 & 0 & 0 & 0 & 0 & 0 & 0 \\
    \hline
  \end{tabular}
\end{center}
Now assume that a tRNA charged with an AA 1 is binded.
\begin{center}
  \tiny
  \begin{tabular}{|c|c|c|c|c|c|c|c|c|c|c|c|}
    \hline
    % after \\: \hline or \cline{col1-col2} \cline{col3-col4} ...
    Aggregation & 30S & 50S & non charged tRNA & tRNA 1 & tRNA 2 & tRNA 3 & tRNA 4 & AA 1 & AA 2 & AA 3 & AA 4 \\
    \hline
         2      &  0 & 0 & 0 & 1 & 0 & 0 & 0 & 0 & 0 & 0 & 0 \\
    \hline
  \end{tabular}
\end{center}
Then elongation is performed. Of course an ATP would be consumed but it is not here to be modeled.
\begin{center}
  \tiny
  \begin{tabular}{|c|c|c|c|c|c|c|c|c|c|c|c|}
    \hline
    % after \\: \hline or \cline{col1-col2} \cline{col3-col4} ...
    Aggregation & 30S & 50S & non charged tRNA & tRNA 1 & tRNA 2 & tRNA 3 & tRNA 4 & AA 1 & AA 2 & AA 3 & AA 4 \\
    \hline
         2      &  0 & 1 & 0 & 0 & 0 & 0 & 0 & 1 & 0 & 0 & 0 \\
    \hline
  \end{tabular}
\end{center}
Now assume that a 30S binds to the binding site, and then a 50S also binds to form a 70S. The state would evolve like follows.
\begin{center}
  \tiny
  \begin{tabular}{|c|c|c|c|c|c|c|c|c|c|c|c|}
    \hline
    % after \\: \hline or \cline{col1-col2} \cline{col3-col4} ...
    Aggregation & 30S & 50S & non charged tRNA & tRNA 1 & tRNA 2 & tRNA 3 & tRNA 4 & AA 1 & AA 2 & AA 3 & AA 4 \\
    \hline
         2      &  0 & 1 & 0 & 0 & 0 & 0 & 0 & 1 & 0 & 0 & 0 \\
         1      &  1 & 0 & 0 & 0 & 0 & 0 & 0 & 0 & 0 & 0 & 0 \\
    \hline
  \end{tabular}
\end{center}
\begin{center}
  \tiny
  \begin{tabular}{|c|c|c|c|c|c|c|c|c|c|c|c|}
    \hline
    % after \\: \hline or \cline{col1-col2} \cline{col3-col4} ...
    Aggregation & 30S & 50S & non charged tRNA & tRNA 1 & tRNA 2 & tRNA 3 & tRNA 4 & AA 1 & AA 2 & AA 3 & AA 4 \\
    \hline
         2      &  0 & 1 & 0 & 0 & 0 & 0 & 0 & 1 & 0 & 0 & 0 \\
         1      &  0 & 1 & 0 & 0 & 0 & 0 & 0 & 0 & 0 & 0 & 0 \\
    \hline
  \end{tabular}
\end{center}
Now we assume that we allow the 70S in aggregation 1 to go to aggregation 2 because we are very happy with this happening (!).
\begin{center}
  \tiny
  \begin{tabular}{|c|c|c|c|c|c|c|c|c|c|c|c|}
    \hline
    % after \\: \hline or \cline{col1-col2} \cline{col3-col4} ...
    Aggregation & 30S & 50S & non charged tRNA & tRNA 1 & tRNA 2 & tRNA 3 & tRNA 4 & AA 1 & AA 2 & AA 3 & AA 4 \\
    \hline
         2      &  0 & 1 & 0 & 0 & 0 & 0 & 0 & 1 & 0 & 0 & 0 \\
         2      &  0 & 1 & 0 & 0 & 0 & 0 & 0 & 0 & 0 & 0 & 0 \\
    \hline
  \end{tabular}
\end{center}
It is possible to be less detailed about the tracking of each ribosome.
