It is assumed here that the formation of the 30S and 50S complexes are described and modeled somewhere else as well as for the metabolites.



\subsubsection{Transfer RNA charging or loading}
%\paragraph{Biological process}
A tRNA has an anticodon which corresponds to a specific amino acid. There are 64 possibilities for a codon for twenty or so amino acids. Let's number the possibilities with $i$. Each one of these possibilities leads to a type of tRNA, $tRNA^i$. The mechanism of charging the tRNA is performed by an enzyme, an aminoacyl tRNA synthetase (aaRS), which is also specific to an anticodon sequence. The mechanism is described below:
$$
  \left\{
    \begin{array}{l}
      \reactionRev{aaRS^i + ATP}{aaRS^i_t}{}{} \\
      \reactionIrr{aaRS^i_t + AA^i}{aaRS^i_{cm}+PP_i+\ce{H+}}{}{} \\
      \reactionIrr{aaRS^i_{cm} + tRNA^i}{aaRS^i + tRNA^i_c + AMP}{}{} \\
    \end{array}
  \right.
$$
with the pyrophosphate anion $PP_i=\ce{P2O7^4-}$. Pyrophosphate anion is unstable in aqueous solution and hydrolyzes into inorganic phosphate $P_i=\ce{HPO4^2-}$. In short:
%\begin{mdframed}[style=MyFrame]
$$
  \left\{
    \begin{array}{l}
    \reactionIrr{AA^i+ATP+tRNA^i}{tRNA^i_c+AMP+PP_i+\ce{H+}}{aaRS^i}{} \\
    \reactionIrr{PP_i + \ce{H2O}}{2P_i}{}{} \\
  \end{array}
  \right. .
$$
%\end{mdframed}
Finally, note that during the elongation, a \ce{H2O} is consumed. If we pair both process, it is possible to write
%\begin{mdframed}[style=MyFrame]
$$
  \reactionIrr{AA^i+ATP+tRNA^i}{tRNA^i_c+AMP+2P_i+\ce{H+}}{aaRS^i}{} \\
$$
%\end{mdframed}



%\paragraph{Biological error} The authors are not aware of any biological error that can happen during tRNA loading.



\subsubsection{Initiation}
%\paragraph{Biological process}
The purpose of the initiation is to locate precisely the initiation $fMet$ amino acid at the Ribosome Binding Site where is located the start codon of the mRNA. The initiation is a complex mechanism that starts from the association of the initiation factors to the $30S$ complex and ends with the formation of the $70S$ complex via the binding of the $50S$ complex to the mRNA. The active steps are illustrated in Figure \ref{fig:initTranslate}.
\begin{figure}[hbtp]
  \centering
  % Requires \usepackage{graphicx}
  \includegraphics[width=15cm]{figure/prokaryoticTranslationInitiationFromWikibooks.png}\\
  \caption{Initiation of translation @Wikipedia}\label{fig:initTranslate}
\end{figure}
The steps are detailed hereafter:
%\begin{enumerate}
%  \item binding of the metabolites of $IF_3$, then $IF_1$ and finally a form $IF_2$ to the $30S$ complex, referred to as $30S_{312}$. The different forms of $IF_2$ are:
%  \begin{enumerate}
%    \item $IF_2$: the initiation factor without cofactor. The corresponding complex is referred to as $30S_{312}$ (it can be assumed that this does never happen),
%    \item $IF_{2a}$: the active form coming from the reaction $\reactionRev{IF_2+GTP}{IF_{2a}}{}{}$. The corresponding complex is referred to as $30S_{312a}$,
%    \item $IF_{2d}$: an inactive form coming from the reaction $\reactionRev{IF_2+GDP}{IF_{2d}}{}{}$. The corresponding complex is referred to as $30S_{312d}$,
%    \item $IF_{2p}$: an inactive form coming from the reaction $\reactionRev{IF_2+ppGpp}{IF_{2p}}{}{}$. The corresponding complex is referred to as $30S_{312p}$;
%  \end{enumerate}
%  \item the $30S$ complex then binds to the mRNA following
%  $$
%    \left\{
%      \begin{array}{l}
%%        \reactionRev{30S_{312}+mRNA}{m30S_{312}}{}{} \\
%        \reactionRev{30S_{312a}+mRNA}{m30S_{312a}}{}{} \\
%        \reactionRev{30S_{312d}+mRNA}{m30S_{312d}}{}{} \\
%        \reactionRev{30S_{312p}+mRNA}{m30S_{312p}}{}{} \\
%      \end{array}
%    \right.
%  $$
%  \item the $m30S_{312a}$ complex has a high affinity with charged tRNA with $fMET$ (denoted by $tRNA_{fMET}$) \textcolor[rgb]{1.00,0.00,0.00}{How does the charging of tRNA works?}. It is assumed that $m30S_{312p}$ does not allow this binding:
%      $$
%      \left\{
%        \begin{array}{l}
%          \reactionIrr{m30S_{312a}+tRNA_{fMET}}{mt30S_{312afMET}}{}{} \\
%          \reactionRev{m30S_{312d}+tRNA_{fMET}}{mt30S_{312dfMET}}{}{} \\
%        \end{array}
%      \right. ;
%      $$
%  \item $IF_3$ is then released:
%        $$
%      \left\{
%        \begin{array}{l}
%          \reactionRev{mt30S_{312afMET}}{mt30S_{12afMET}+IF_3}{}{} \\
%          \reactionRev{mt30S_{312dfMET}}{mt30S_{12dfMET}+IF_3}{}{} \\
%        \end{array}
%      \right. ;
%      $$
%  \item the $50S$ complex binds to form the $70S$ complex and $IF_1$ and $IF_2$ are released. Only an active form allows this binding:
%  $$
%    \reactionIrr{mt30S_{12afMET}+50S}{70S+IF_1+IF_{2d}}{}{}.
%  $$
%\end{enumerate}
%\noindent Note that the $tRNA_{fMET}$ is already at P-site so that the complex is ready for elongation.
\begin{enumerate}
  \item in parallel, the $30S$ complex binds to the $mRNA$ when the $IF_3$ and $IF_1$ have already binded to it whereas $IF_2$ binds with a co-factor and to a $tRNA^{fMet}$ (an $fMet$ charged tRNA):
    \begin{itemize}
    \item the $30S$ complex binds to the $mRNA$:
      \begin{enumerate}
      \item the binding of the initiation factors $IF_3$ and $IF_1$ to a $30S$ complex:
        $$
        \left\{
          \begin{array}{l}
          \reactionRev{30S+IF_3}{30S_3}{}{} \\
          \reactionRev{30S_3+IF_1}{30S_{31}}{}{} \\
          \end{array}
        \right.                                                                                                                                                 ,
        $$
      \item the binding of formed complex to the $mRNA$:
        $$
        \reactionRev{30S_{31}+mRNA}{m30S_{31}}{}{}{}
        $$
      \end{enumerate}
    \item the binding of the initiation factor $IF_2$ with a cofactor and to a $tRNA^{fMet}$:
      \begin{itemize}
      \item $IF_2$: the initiation factor without cofactor. It is assumed that $IF_2$ does not bind with a $tRNA^{fMet}$,
%        $$
%        \reactionRev{tRNA^{fMet}+IF_{2}}{tRNA_{2}^{fMet}}{}{} \\
%        $$
      \item $IF_{2a}$: the active form coming from the $GTP$ cofactor
        $$
        \left\{
          \begin{array}{l}
          \reactionRev{IF_2+GTP}{IF_{2a}}{}{} \\
          \reactionRev{tRNA^{fMet}+IF_{2a}}{tRNA_{2a}^{fMet}}{}{} \\
          \end{array}
        \right. ,
        $$
      \item $IF_{2d}$: an inactive form coming from the $GDP$ cofactor
        $$
        \left\{
          \begin{array}{l}
          \reactionRev{IF_2+GDP}{IF_{2d}}{}{} \\
          \reactionRev{tRNA^{fMet}+IF_{2d}}{tRNA_{2d}^{fMet}}{}{} \\
          \end{array}
        \right. ,
        $$
      \item $IF_{2p}$: an inactive form coming from the $ppGpp$ cofactor
        $$
        \left\{
          \begin{array}{l}
          \reactionRev{IF_2+ppGpp}{IF_{2p}}{}{} \\
          \reactionRev{tRNA^{fMet}+IF_{2p}}{tRNA_{2p}^{fMet}}{}{} \\
          \end{array}
        \right. ;
        $$
      \end{itemize}
    \end{itemize}
  \item the $m30S_{31}$ complex has a high affinity with charged tRNA with $fMet$ (denoted by $tRNA^{fMet}$). It is assumed that $tRNA_{2p}^{fMet}$ does not binds:
    $$
    \left\{
      \begin{array}{l}
      \reactionIrr{m30S_{31}+tRNA_{2a}^{fMet}}{mt30S_{312a}^{fMet}}{}{} \\
      \reactionRev{m30S_{31}+tRNA_{2d}^{fMet}}{mt30S_{312d}^{fMet}}{}{} \\
      \end{array}
    \right. ;
    $$
  \item $IF_3$ is then released:
    $$
    \reactionIrr{mt30S_{312a}^{fMet}}{mt30S_{12a}^{fMet}+IF_3}{}{} ;
    $$
%    $$
%    \left\{
%      \begin{array}{l}
%      \reactionIrr{mt30S_{312a}^{fMet}}{mt30S_{12a}^{fMet}+IF_3}{}{} \\
%      \reactionRev{mt30S_{312d}^{fMet}}{mt30S_{12d}^{fMet}+IF_3}{}{} \\
%      \end{array}
%    \right. ;
%    $$
  \item the $50S$ complex binds to form the $70S$ complex and $IF_1$ and $IF_2$ are released. Only an active form allows this binding:
    $$
    \reactionIrr{mt30S_{12a}^{fMet}+50S}{70S+IF_1+IF_{2d}+P_i+\ce{H2O}}{}{}.
    $$
\end{enumerate}
\noindent Note that the $tRNA^{fMet}$ is already at P-site of the $70S$ complex so that it is ready for elongation.

\medskip

In summary, we have the possible overall reactions:
%\begin{mdframed}[style=MyFrame]
$$
  \left\{
    \begin{array}{l}
      \reactionRev{30S+IF_3+IF_1+mRNA}{m30S_{31}}{}{} \\
      \reactionRev{IF_2+GTP}{IF_{2a}}{}{} \\
      \reactionRev{IF_2+GDP}{IF_{2d}}{}{} \\
      \reactionRev{IF_2+ppGpp}{IF_{2p}}{}{} \\
      \reactionRev{IF_{2p}+tRNA^{fMet}}{tRNA_{2p}^{fMet}}{}{}{} \\
      \reactionRev{m30S_{31}+IF_{2d}+tRNA^{fMet}}{mt30S_{312d}^{fMet}}{}{}{} \\
      \reactionIrr{m30S_{31}+IF_{2a}+tRNA^{fMet}}{70S+IF_1+IF_{2d}+IF_3+P_i+\ce{H2O}}{}{}{} \\
    \end{array}
  \right.
$$
%\end{mdframed}



%\paragraph{Biological error} The authors are not aware of any biological error that can happen during translation initiation.



\subsubsection{Elongation}
%\paragraph{Biological process}
Elongation is the process of creating the protein by putting amino acid one by one. Roughly, the elongation is illustrated in Figure \ref{fig:elongationTranslate}
\begin{figure}[hbtp]
  \centering
  % Requires \usepackage{graphicx}
  \includegraphics[width=9cm]{figure/proteinTranslationFromWikipediaModif.png}\\
  \caption{Elongation during translation @Wikipedia}\label{fig:elongationTranslate}
\end{figure}
and follows the steps:
\begin{enumerate}
  \item a charged tRNA binds to the A-site of the 70S complex;
  \item a peptide bond is formed between the new amino acid and the already here one at P-site;
  \item the ribosome translocate and the tRNA at E-site is released.
\end{enumerate}
The steps are explained hereafter, with the current (at A-site) codon being $i_0$:
\begin{enumerate}
  \item a tRNA (charged or uncharged) binds to the A-site of the 70S ribosomal complex. The affinity for cognate tRNA is higher the one for near-cognate tRNA. This tRNA is carried by the GTPase EF-Tu whose hydrolysis allows for the decoding between the mRNA codon and the tRNA anti-codon on the A-site. EF-Tu is released:
      \begin{enumerate}
        \item an EF-Tu with GTP ($EFTu_a$) binds to the tRNA (charged or uncharged)
          $$
            \left\{
            \begin{array}{l}
              \reactionRev{tRNA^i_c + EFTu_a}{EtRNA^i_{ca}}{}{} \\
              \reactionRev{tRNA^i + EFTu_a}{EtRNA^i_{a}}{}{} \\
            \end{array}
          \right.
          $$
        \item this tRNA then binds to the A-site of the 70S complex and EF-Tu is released (with GDP denoted $EFTu_d$)
          $$
            \left\{
            \begin{array}{l}
              \reactionRev{EtRNA^i_{a} + 70S}{Et70S^i_{a}}{}{} \\
              \reactionIrr{EtRNA^i_{ca} + 70S +\ce{H2O}}{t70S^i_{c} + EFTu_d + P_i}{}{} \\
            \end{array}
          \right.
          $$
          It s possible that the charged tRNA that comes to the A-site does not correspond to the anti-codon of the mRNA. The protein produced may still functional due to the redundance of the amino acid coding, or that amino acid location is not crucial for the folding and function of the protein;
          \textcolor[rgb]{1.00,0.00,0.00}{Proofreading (Olivier PhD)}
        \item as an aside, $EFTu_d$ is transformed back to $EFTu_a$ via a GTPase EF-Ts:
          $$
            \reactionIrr{EFTu_d + GTP}{EFTu_a + GDP}{EFTs}{}
          $$
      \end{enumerate}
  \item the peptide bond is formed via the elongation factor EF-P
    $$
      \left\{
        \begin{array}{l}
          \reactionRev{EFP + GTP}{EFP_a}{}{} \\
          \reactionIrr{t70S^i_{c} + EFP_a}{t70S^{+i} + EFP_d + \ce{H2O} + P_i}{}{} \\
        \end{array}
      \right.
    $$
    with $EFP_d$ being the GDP bounded form of $EFP$.
  \item translocation takes place via a elongation factor EF-G, a GTPase:
    $$
      \reactionIrr{t70S^{+i} + GTP+\ce{H2O}}{70S^{+} + tRNA^i + GDP + P_i}{EFG}{}.
    $$
%    EF-G is normally inhibited by fusidic acid, but resistance has emerged.
\end{enumerate}



%\paragraph{Biological error} It s possible that the charged tRNA that comes to the A-site does not correspond to the anti-codon of the mRNA, that is $i\neq i_0$. It is possible that the protein produced is still functional due to the redundance of the amino acid coding, or that amino acid location is not crucial for the folding and function of the protein.


\subsubsection{Termination}
%\paragraph{Biological process}
When the ribosome encounters a stop codon, it release the mRNA and the protein. In more details, when after the translocation step of elongation the ribosome encounters at its A-site a stop codon, a release factor (either $RF_1$ or $RF_2$ depending on the stop codon) binds. Note that tRNA are unable to recognize a stop codon. A final GTPase $RF_3$ hydrolysis allows the release of the protein. It can be summarized as follows:
$$
  \left\{
    \begin{array}{l}
      \reactionRev{RF_3 + GTP}{RF_{3a}}{}{} \\
      \reactionRev{70S + RF_1}{70S_1}{}{} \\
      \reactionRev{70S + RF_2}{70S_2}{}{} \\
      \reactionIrr{70S_1 + RF_{3a} + \ce{H2O}}{30S + 50S + protein + RF_1 + RF_{3d} + P_i}{}{} \\
      \reactionIrr{70S_2 + RF_{3a} + \ce{H2O}}{30S + 50S + protein + RF_2 + RF_{3d} + P_i}{}{} \\
    \end{array}
  \right.
$$
with $RF_{3d}$ being GDP bounded form of $RF_3$.

\textcolor[rgb]{1.00,0.00,0.00}{From Olivier PhD, last AA not in the protein.}



\subsubsection{Translation termination on a broken mRNA} \label{sec:stalledTranslation}

\begin{figure}[hbtp]
  \centering
  % Requires \usepackage{graphicx}
  \includegraphics[width=15cm]{figure/tmRNAProcess.png}\\
  \caption{Termination on stalled ribosome @Wikipedia}\label{fig:translationStalledRibosome}
\end{figure}
tmRNA coded by ssrA.

%\paragraph{Biological error} The authors are not aware of any biological error that can happen during translation termination.

